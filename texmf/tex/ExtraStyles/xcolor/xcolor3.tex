%%
%% This is file `xcolor3.tex',
%% generated with the docstrip utility.
%%
%% The original source files were:
%%
%% xcolor.dtx  (with options: `test3')
%% 
%% IMPORTANT NOTICE:
%% 
%% For the copyright see the source file.
%% 
%% Any modified versions of this file must be renamed
%% with new filenames distinct from xcolor3.tex.
%% 
%% For distribution of the original source see the terms
%% for copying and modification in the file xcolor.dtx.
%% 
%% This generated file may be distributed as long as the
%% original source files, as listed above, are part of the
%% same distribution. (The sources need not necessarily be
%% in the same archive or directory.)
%%
%% ---------------------------------------------------------
%% Copyright (C) 2003-2004 by Dr. Uwe Kern <xcolor@ukern.de>
%% ---------------------------------------------------------
%%
%% Please send error reports and suggestions for
%% improvements to the above email address.
%%
%%
\def\XCfilesource{xcolor.dtx}%
\def\XCfileversion{v2.00}%
\def\XCfiledate{2004/07/04}%
\listfiles
\ProvidesFile{xcolor3}[\XCfiledate\space\XCfileversion\space
                       Color logging test (UK)]

\documentclass{article}
\usepackage[table,dvipsnames]{xcolor}[2004/07/04]

\tracingcolors=4
%%\tracingcolors=3
%%\tracingcolors=2
%%\tracingcolors=1
%%\tracingcolors=0

\parindent0pt
\pagecolor{gray!25}

\begin{document}
\title{Color extensions with the \textsf{xcolor} package --- log file example}
\author{Dr. Uwe Kern\thanks{This file is part of the \textsf{xcolor} distribution which can be downloaded from the CTAN mirrors (\texttt{macros/latex/contrib/xcolor/}) or the homepage \texttt{www.ukern.de/tex/xcolor.html}. Please send error reports and suggestions for improvements to \texttt{xcolor@ukern.de}.}}
\date{\XCfileversion{} (\XCfiledate)}
\maketitle

The purpose of this file is to demonstrate the logging facilities of the \textsf{xcolor} package.
By playing around with different values of \texttt{\string\tracingcolors}, one can observe the different behavior in the \texttt{log} file.

\bigskip
Table example:
\rowcolors[\hline]{1}{green!25}{yellow!50}
\begin{tabular}{ll}
test & row \number\rownum\\
test & row \number\rownum\\
\rowcolor{blue!25}
test & row \number\rownum\\
test & row \number\rownum\\
\hiderowcolors
test & row \number\rownum\\
test & row \number\rownum\\
\showrowcolors
test & row \number\rownum\\
test & row \number\rownum\\
\multicolumn{1}%
 {>{\columncolor{red!12}}l}{test} & row \number\rownum\\
\end{tabular}

\bigskip
\providecolor{dummy}{rgb}{.6,.5,.4}
\definecolor{dummy}{rgb}{.6,.5,.4}
\providecolor{dummy}{rgb}{.6,.5,.4}
\hbox{\textcolor{dummy}{Test with \texttt{\string\definecolor}}}

{\color[rgb]{.4,.5,.6}Test with \texttt{\string\color}}

\bigskip
Current color application:\par
\def\test{current, \textcolor{.!50}{50\%}, \textcolor{-.}{complement},
          \textcolor{yellow!50!.}{mix}}
\textcolor{blue}{\test} and \textcolor{red}{\test},\par
\def\Test{\color{.!80}Test}
\textcolor{blue}{\Test\Test\Test\Test\Test} and
\textcolor{red}{\Test\Test\Test\Test\Test}.

\bigskip
Current color test with \texttt{\string\definecolorseries}:\par
\color{blue}
\definecolorseries{foo}{rgb}{last}{.}{-.}
\resetcolorseries[5]{foo}
\def\test{\hbox to 1em{{\color{foo!!+}\vrule width 1em height 1.5ex}}}
Test\test\test\test\test\test\test Test

\resetcolorseries[5]{foo}
\def\test{\hbox to 1em{{\color{foo!!++}\vrule width 1em height 1.5ex}}}
Test\test\test\test\test\test\test Test

\resetcolorseries[5]{foo}
\def\test{\hbox to 1em{{\color{foo!![2]}\vrule width 1em height 1.5ex}}}
Test\test\test\test\test\test\test Test

\bigskip
\color{black}
Test with named colors:\par
\color{blue}
Test: \textcolor[named]{JungleGreen}{JungleGreen};
Test: \textcolor{JungleGreen}{JungleGreen};
Test: \textcolor{JungleGreen!50!DarkOrchid}{JungleGreen!50!DarkOrchid};
Test: \textcolor{green!50!red}{green!50!red}.

Type test:
\makeatletter
\@namedef{\string\color@foo1}{foo1{}{}{}{}}\edef\tempa{\XC@type{foo1}}\tempa
\@namedef{\string\color@foo2}{\xcolor@{foo2}{}{}{}}\edef\tempb{\XC@type{foo2}}\tempb
\@namedef{\string\color@foo3}{\xcolor@{}{foo3}{}{}}\edef\tempc{\XC@type{foo3}}\tempc
\@namedef{\string\color@foo4}{\xcolor@{}{}{foo4}{}}\edef\tempd{\XC@type{foo4}}\tempd
\makeatother

\textcolor{rgb:red!50,4;green!25,2}{Extended color expression (rgb:red!50,4;green!25,2)}.

\end{document}
\endinput
%%
%% End of file `xcolor3.tex'.
