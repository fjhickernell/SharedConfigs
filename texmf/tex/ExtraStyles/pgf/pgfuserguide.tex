\documentclass{ltxdoc}

% Copyright 2003 by Till Tantau <tantau@cs.tu-berlin.de>.
%
% This program can be redistributed and/or modified under the terms
% of the LaTeX Project Public License Distributed from CTAN
% archives in directory macros/latex/base/lppl.txt.

\usepackage{pgf,pgfarrows,pgfautomata,pgfheaps,pgfnodes,pgfshade}
\usepackage[left=2.25cm,right=2.25cm,top=2.5cm,bottom=2.5cm,nohead]{geometry}
\usepackage{amsmath,amssymb}
\usepackage[pdfborder={0 0 0}]{hyperref}
\usepackage{xxcolor}

\def\pgf{\textsc{pgf}}
\def\pstricks{\textsc{pstricks}}
\def\Class#1{\hbox{\small#1}}
\def\bs{$\backslash$}

\def\Environment#1{\par\bigskip\noindent\textbf{Environment \texttt{#1}}\par}
\def\Command#1{\par\bigskip\noindent\textbf{Command \texttt{#1}}\par}
\long\def\Parameters#1{\medskip\noindent Parameters:
  \begin{enumerate}\itemsep=0pt\parskip=0pt
    #1
  \end{enumerate}}
\long\def\Description#1{\unskip\medskip\noindent Description: #1}
\def\Example{\par\medskip\noindent Example: }


\renewcommand*\descriptionlabel[1]{\hspace\labelsep\normalfont #1}


\def\declare#1{{\color{red!75!black}#1}}

\def\command#1{\list{}{\leftmargin=2em\itemindent-\leftmargin\def\makelabel##1{\hss##1}}%
\item\extractcommand#1@\par\topsep=0pt}
\def\endcommand{\endlist}
\def\extractcommand#1#2@{\strut\declare{\texttt{\string#1}}#2}

\def\example{\par\smallskip\noindent\textit{Example: }}

\def\environment#1{\list{}{\leftmargin=2em\itemindent-\leftmargin\def\makelabel##1{\hss##1}}%
\extractenvironement#1@\par\topsep=0pt}
\def\endenvironment{\endlist}
\def\extractenvironement#1#2@{%
\item{{\ttfamily\char`\\begin\char`\{\declare{#1}\char`\}}#2}%
  {\itemsep=0pt\parskip=0pt\item{\meta{environment contents}}%
  \item{\ttfamily\char`\\end\char`\{\declare{#1}\char`\}}}}


\def\packageoption#1{\list{}{\leftmargin=2em\itemindent-\leftmargin\def\makelabel##1{\hss##1}}%
\item{{\ttfamily\char`\\usepackage[\declare{#1}]\char`\{pgf\char`\}}}\par\topsep=0pt}
\def\endpackageoption{\endlist}

\def\smallpackage{\vbox\bgroup\package}
\def\endsmallpackage{\egroup\endpackage}
\def\package#1{\list{}{\leftmargin=2em\itemindent-\leftmargin\def\makelabel##1{\hss##1}}%
\extracttheme#1@\par\topsep=0pt}
\def\endpackage{\endlist}
\def\extracttheme#1#2@{%
\item{{{\ttfamily\char`\\usepackage}#2{\ttfamily\char`\{\declare{#1}\char`\}}}}}


\newcommand\opt[1]{{\color{black!50!green}#1}}
\renewcommand\oarg[1]{\opt{{\ttfamily[}\meta{#1}{\ttfamily]}}}
\newcommand\ooarg[1]{{\ttfamily[}\meta{#1}{\ttfamily]}}
\newcommand\sarg[1]{\opt{{\ttfamily\char`\<}\meta{#1}{\ttfamily\char`\>}}}
\newcommand\ssarg[1]{{\ttfamily\char`\<}\meta{#1}{\ttfamily\char`\>}}

\begin{document}

\title{User's Guide to the PGF Package, Version 0.62\\
  \Large\href{http://www.ctan.org/tex-archive/graphics/pgf/}{|http://www.ctan.org/tex-archive/graphics/pgf/|}}
\author{Till Tantau\\
  \href{mailto:tantau@cs.tu-berlin.de}{|tantau@cs.tu-berlin.de|}}


\maketitle

\tableofcontents

\section{Introduction}

\subsection{Overview}

This user's guide explains the functionality of the \pgf\ package.
\pgf\ stands for `portable graphics format'. It is a \TeX\ macro
package that allows you to create graphics in your \TeX\ documents
using a special |pgfpicture| environment and special macros for
drawing lines, curves, rectangles, and many other kind of graphic
objects. Its usage closely resembles the \pstricks\ package or the
normal picture environment of \LaTeX.

Although \pgf\ is less powerful than \pstricks, it has the advantage
that it can generate both PostScript output and \textsc{pdf} output
from the same file. The \pgf\ package works together both with
|dvips| and |pdftex|. In particular, packages that rely
on |pdftex| or |pdflatex| (like some packages for
creating presentations) can be used together with \pgf.

The package consists of the core style |pgf.sty| and a number
of extension styles that are build on top of it. Currently, the
documented ones are
\begin{itemize}\itemsep=0pt\parskip=0pt
\item
  |pgfarrows.sty|, used to draw a large variety of arrows.
\item
  |pgfnodes.sty|, used to draw nodes in diagrams and to connect
  them in a convenient way.
\item
  |pgfshade.sty|, used to create shadings (also called
  gradients). 
\end{itemize}

In order to use \pgf\ you will have to include the command
\begin{verbatim}
\usepackage{pgf}
\end{verbatim}
at the beginning of your main \TeX\ file. If you also wish to use the
extensions, you also have to include them. For example, you will
typically use the following command:
\begin{verbatim}
\usepackage{pgf,pgfarrows,pgfnodes}
\end{verbatim}

In this guide you will find the descriptions of all ``public''
commands provided by the |pgf| package. In each such description, the
described command, environment or option is printed in red. Text shown
in green is optional and can be left out. Note that (currently) many
commands take arguments in square brackets that are \emph{not}
optional. In some future version of \pgf\ it will possible to omit
these optional arguments. 




\subsection{Installation}

To use \pgf, you just need to put all files with the ending
|.sty| of the pgf package in a directory that is read by
\TeX. You need to have the package |xcolor| installed, version 2.00 or
higher. To uninstall |pgf|, simply remove these files once more. 
Unfortunately, there are different ways of making \TeX\ ``aware'' of
files. Which way you should choose depends on how permanently you
intend to use it.


\subsection{Installing Prebundled Packages}

I do not create or manage prebundled packages of \pgf, but,
fortunately, nice other people do. I cannot give detailed instructions
on how to install these packages, since I do not manage them, but I
\emph{can} tell you were to find them. You install them the ``usual
way'' you install packages. If anyone has any hints and additional
information on this, please email me.

For Debian, you need the packages at
\begin{verbatim}
http://packages.debian.org/pgf
http://packages.debian.org/latex-xcolor
\end{verbatim}

For MiK\TeX, you need the packages called |pgf| and |xcolor|. 



\subsubsection{Temporary Installation}

If you only wish to install \pgf\ for a quick appraisal, do
the following: Obtain all files from the directory
\href{http://www.ctan.org/tex-archive/graphics/pgf/}{|http://www.ctan.org/tex-archive/graphics/pgf/|}.
(most likely, you have already done this).
Place all files in a new directory. For example,
|/home/tantau/pgf/| would work fine for me. Then setup the
environment variable called |TEXINPUTS| to be the following
string (how exactly this is done  depends on your operating system and
shell):

\begin{verbatim}
.:/home/tantau/pgf:
\end{verbatim}

Naturally, if the |TEXINPUTS| variable is already defined
differently, you should \emph{add} the directories to the list. Do
not forget to place a colon at the end (corresponding to an empty
path), which will include all standard directories.



\subsubsection{Installation in a texmf Tree}

For a more permanent installation, you can place the files of the
the \textsc{pgf} package (see the previous subsection on how to obtain
them) in an appropriate |texmf| tree. 

When you ask \TeX\ to use a certain class or package, it usually looks
for the necessary files in so-called |texmf| trees. These trees
are simply huge directories that contain these files. By default,
\TeX\ looks for files in three different |texmf| trees:
\begin{itemize}
\item
  The root |texmf| tree, which is usually located at
  |/usr/share/|, \verb!c:\texmf\!, or\\
  \verb!c:\Program Files\TeXLive\texmf\!.
\item
  The local  |texmf| tree, which is usually located at
  |/usr/local/share/|, \verb!c:\localtexmf\!, or\\
  \verb!c:\Program Files\TeXLive\texmf-local\!.
\item
  Your personal  |texmf| tree, which is located in your home
  directory.   
\end{itemize}

You should install the packages either in the local tree or in
your personal tree, depending on whether you have write access to the
local tree. Installation in the root tree can cause problems, since an
update of the whole \TeX\ installation will replace this whole tree.

Inside whatever |texmf| directory you have chosen, create
the sub-sub-sub-directory |texmf/tex/latex/pgf|
and place all files in it. Then rebuild \TeX's filename database. This
done by running the command  |texhash| or |mktexlsr|
(they are the same). In MikTeX, there is a menu option to do this.

If you want to be really tidy, you can place the documentation in the
directory |texmf/doc/latex/pgf|.

\vskip1em
For a more detailed explanation of the standard installation process
of packages, you might wish to consult
\href{http://www.ctan.org/installationadvice/}{|http://www.ctan.org/installationadvice/|}.
However, note that the \pgf\ package does not come with a
|.ins| file (simply skip that part).




\subsection{Quick Start}

This section presents some simple examples. By copying these examples
and modifying them slightly, you can create your first pictures using
\pgf.

The first example draws a rectangle and a circle next to each other.

\noindent
\begin{pgfpicture}{0cm}{0cm}{5cm}{2cm}
  % (0cm,0cm) is the lower left corner,
  % (5cm,2cm) is the upper right corner.

  \pgfrect[stroke]{\pgfpoint{0cm}{0cm}}{\pgfpoint{2cm}{10pt}}
  % Paint a rectangle (stroke it, do not fill it)
  % The lower left corner is at (0cm,0cm)
  % The rectangle is 2cm wide and 10pt high.

  \pgfcircle[fill]{\pgfpoint{3cm}{1cm}}{10pt}
  % Paint a filled circle
  % The center is at (3cm,1cm)
  % The radius is 10pt
\end{pgfpicture}
\begin{verbatim}
\begin{pgfpicture}{0cm}{0cm}{5cm}{2cm}
  % (0cm,0cm) is the lower left corner,
  % (5cm,2cm) is the upper right corner.

  \pgfrect[stroke]{\pgfpoint{0cm}{0cm}}{\pgfpoint{2cm}{10pt}}
  % Paint a rectangle (stroke it, do not fill it)
  % The lower left corner is at (0cm,0cm)
  % The rectangle is 2cm wide and 10pt high.

  \pgfcircle[fill]{\pgfpoint{3cm}{1cm}}{10pt}
  % Paint a filled circle
  % The center is at (3cm,1cm)
  % The radius is 10pt
\end{pgfpicture}
\end{verbatim}

The |\pgfpoint| command is used to specify a point. It is often
more convenient to use the command |pgfxy| instead, which lets
you specify a point in terms of multiples of a $x$-vector
and a $y$-vector. They are predefined to |\pgfpoint{1cm}{0cm}|
and |\pgfpoint{0cm}{1cm}|, but you can change these settings.

\noindent
\begin{pgfpicture}{0cm}{0cm}{5cm}{1.25cm}

  \pgfline{\pgfxy(0,0)}{\pgfxy(1,1)}
  % Draws a line from (0cm,0cm) to (1cm,1cm)
  % Command \pgfline{\pgfpoint{0cm}{0cm}}{\pgfpoint{1cm}{1cm}}
  % would have the same effect.
  
  \pgfline{\pgfxy(1,1)}{\pgfxy(2,1)}
  \pgfline{\pgfxy(2,1)}{\pgfxy(3,0)}
  \pgfline{\pgfxy(3,0)}{\pgfxy(4,0)}
  \pgfline{\pgfxy(4,0)}{\pgfxy(5,1)}
\end{pgfpicture}
\begin{verbatim}
\begin{pgfpicture}{0cm}{0cm}{5cm}{1.25cm}

  \pgfline{\pgfxy(0,0)}{\pgfxy(1,1)}
  % Draws a line from (0cm,0cm) to (1cm,1cm)
  % Command \pgfline{\pgfpoint{0cm}{0cm}}{\pgfpoint{1cm}{1cm}}
  % would have the same effect.
  
  \pgfline{\pgfxy(1,1)}{\pgfxy(2,1)}
  \pgfline{\pgfxy(2,1)}{\pgfxy(3,0)}
  \pgfline{\pgfxy(3,0)}{\pgfxy(4,0)}
  \pgfline{\pgfxy(4,0)}{\pgfxy(5,1)}
\end{pgfpicture}
\end{verbatim}

You can put text into a picture using the |\pgfbox| command.

\noindent
\begin{pgfpicture}{0cm}{0cm}{5cm}{2cm}
  \pgfputat{\pgfxy(1,1)}{\pgfbox[center,center]{Hi!}}
  % pgfputat places something at a certain position
  % pgfbox shows the text `hi!'. The horizontal alignment
  % is centered (other options: left, right). The vertical
  % alignment is also centered (other options: top, bottom,
  % base).

  \pgfcircle[stroke]{\pgfxy(1,1)}{0.5cm}

  \pgfsetendarrow{\pgfarrowto}
  % In the following, all lines will end with an arrow that looks like
  % the arrow of TeX's \to command

  \pgfline{\pgfxy(1.5,1)}{\pgfxy(2.2,1)}

  \pgfputat{\pgfxy(3,1)}{
    \begin{pgfrotateby}{\pgfdegree{30}}
      % You can rotate things like this
      \pgfbox[center,center]{$\int_0^\infty xdx$}
    \end{pgfrotateby}}
  \pgfcircle[stroke]{\pgfxy(3,1)}{0.75cm}
\end{pgfpicture}
\begin{verbatim}
\begin{pgfpicture}{0cm}{0cm}{5cm}{2cm}
  \pgfputat{\pgfxy(1,1)}{\pgfbox[center,center]{Hi!}}
  % pgfputat places something at a certain position
  % pgfbox shows the text `hi!'. The horizontal alignment
  % is centered (other options: left, right). The vertical
  % alignment is also centered (other options: top, bottom,
  % base).

  \pgfcircle[stroke]{\pgfxy(1,1)}{0.5cm}

  \pgfsetendarrow{\pgfarrowto}
  % In the following, all lines will end with an arrow that looks like
  % the arrow of TeX's \to command

  \pgfline{\pgfxy(1.5,1)}{\pgfxy(2.2,1)}

  \pgfputat{\pgfxy(3,1)}{
    \begin{pgfrotateby}{\pgfdegree{30}}
      % You can rotate things like this
      \pgfbox[center,center]{$\int_0^\infty xdx$}
    \end{pgfrotateby}}
  \pgfcircle[stroke]{\pgfxy(3,1)}{0.75cm}
\end{pgfpicture}
\end{verbatim}

In order to draw curves and complicated lines, you can use the commands
|pgfmoveto|, |pgflineto|, and |pgfcurveto|. To
actually draw or fill the painted area, you use |pgfstroke| or
|pgffill|.

\noindent
\begin{pgfpicture}{0cm}{0cm}{5cm}{2cm}
  \pgfmoveto{\pgfxy(0,1)}
  \pgfcurveto{\pgfxy(1,0.5)}{\pgfxy(1,1.5)}{\pgfxy(2,1)}
  \pgfstroke

  \pgfsetdash{{3pt}{3pt}}{0pt}
  \pgfmoveto{\pgfxy(0,1)}
  \pgflineto{\pgfxy(1,0.5)}
  \pgflineto{\pgfxy(1,1.5)}
  \pgflineto{\pgfxy(2,1)}
  \pgfstroke

  \pgfmoveto{\pgfxy(3,1)}
  \pgfcurveto{\pgfxy(3,0)}{\pgfxy(4,0)}{\pgfxy(4,1)}
  \pgfcurveto{\pgfxy(4,2)}{\pgfxy(3,2)}{\pgfxy(3,1)}
  \pgfclosepath
  \pgffill  
\end{pgfpicture}
\begin{verbatim}
\begin{pgfpicture}{0cm}{0cm}{5cm}{2cm}
  \pgfmoveto{\pgfxy(0,1)}
  \pgfcurveto{\pgfxy(1,0.5)}{\pgfxy(1,1.5)}{\pgfxy(2,1)}
  \pgfstroke

  \pgfsetdash{{3pt}{3pt}}{0pt}
  \pgfmoveto{\pgfxy(0,1)}
  \pgflineto{\pgfxy(1,0.5)}
  \pgflineto{\pgfxy(1,1.5)}
  \pgflineto{\pgfxy(2,1)}
  \pgfstroke

  \pgfmoveto{\pgfxy(3,1)}
  \pgfcurveto{\pgfxy(3,0)}{\pgfxy(4,0)}{\pgfxy(4,1)}
  \pgfcurveto{\pgfxy(4,2)}{\pgfxy(3,2)}{\pgfxy(3,1)}
  \pgfclosepath
  \pgffill  
\end{pgfpicture}
\end{verbatim}


\subsection{Gallery}

In the following, a number of figures are shown that have been created
using \pgf. Please see the source code for how they are created.

\pgfdeclareverticalshading{shadow}{200pt}{%
  rgb(0pt)=(.2,.2,.2);
  rgb(110pt)=(1,1,1)}
\pgfdeclareverticalshading{paper}{200pt}{%
  rgb(0pt)=(0.8,0.8,0.5);
  rgb(150pt)=(1,1,1)}
\pgfdeclareverticalshading{pic}{25pt}{%
  rgb(0pt)=(0.25,0.75,0.25);
  rgb(15pt)=(0.75,0.25,0.25);
  rgb(35pt)=(0.25,0.25,0.75)}
\pgfdeclareverticalshading{corner}{20pt}{%
  rgb(0pt)=(0.5,0.5,0);
  rgb(20pt)=(0.8,0.8,0.8)}
\pgfdeclarehorizontalshading{cover}{200pt}{%
  rgb(0pt)=(0.84,.5,.5);
  rgb(18pt)=(0.82,.48,.48);
  rgb(19pt)=(0.83,.66,.65);
  rgb(21pt)=(0.83,.66,.65);
  rgb(30pt)=(0.69,.25,.3);
  rgb(80pt)=(0.45,0.05,0.05)}
\pgfdeclareverticalshading{side}{100pt}{%
  rgb(0pt)=(0.78,.78,.78);
  rgb(25pt)=(0.60,.60,.60);
  rgb(50pt)=(0.25,.25,.25)}


\begin{pgfpicture}{-10pt}{-20pt}{100pt}{120pt}
  \pgfsetxvec{\pgfpoint{10pt}{0pt}}
  \pgfsetyvec{\pgfpoint{0pt}{10pt}}
  \pgfsetlinewidth{4pt}

  \begin{pgfscope}
    \pgfmoveto{\pgfxy(0,0)}
    \pgflineto{\pgfxy(8,0)}
    \pgflineto{\pgfxy(8,9)}
    \pgflineto{\pgfxy(6,9)}
    \pgflineto{\pgfxy(6,11)}
    \pgflineto{\pgfxy(0,11)}
    \pgfclip

    \pgfputat{\pgfxy(0,-10)}
    {%
      \begin{pgfrotateby}{\pgfdegree{45}}
        \pgfbox[left,base]{\pgfuseshading{paper}}
      \end{pgfrotateby}
    }
  \end{pgfscope}
  
  \begin{pgfscope}
    \pgfmoveto{\pgfxy(8,9)}
    \pgflineto{\pgfxy(6,9)}
    \pgflineto{\pgfxy(6,11)}
    \pgfclip

    \pgfputat{\pgfxy(6,9)}{\pgfbox[left,base]{\pgfuseshading{corner}}}
  \end{pgfscope}

  \pgfmoveto{\pgfxy(0,0)}
  \pgflineto{\pgfxy(8,0)}
  \pgflineto{\pgfxy(8,9)}
  \pgflineto{\pgfxy(6,11)}
  \pgflineto{\pgfxy(0,11)}
  \pgfclosepath
  \pgfstroke
  
  \color[gray]{0.5}
  \pgfxyline(1,9.5)(6,9.5)
  \color[gray]{0.6}
  \pgfxyline(2,8)(6,8)
  \pgfxyline(2,7)(6,7)
  
  \color[gray]{0.7}
  \pgfxyline(1,5.5)(3.5,5.5)
  \pgfxyline(1,4.5)(3.5,4.5)
  \pgfxyline(1,3.5)(3.5,3.5)
  \pgfxyline(1,2.5)(3.5,2.5)
  \pgfxyline(1,1.5)(3.5,1.5)

  \pgfputat{\pgfxy(4.5,2.25)}{\pgfbox[left,base]{\pgfuseshading{pic}}}
  \pgfxyline(4.5,1.5)(7,1.5)

  \color{black}
  \pgfmoveto{\pgfxy(8,9)}
  \pgflineto{\pgfxy(6,9)}
  \pgflineto{\pgfxy(6,11)}
  \pgfstroke
\end{pgfpicture}
\begin{pgfpicture}{0pt}{0pt}{140pt}{110pt}
  \pgfsetxvec{\pgfpoint{10pt}{0pt}}
  \pgfsetyvec{\pgfpoint{0pt}{10pt}}
  \pgfsetlinewidth{4pt}
  \pgfsetroundjoin
  
  \pgfsetlinewidth{8pt}
  \color[gray]{0.5}
  \pgfmoveto{\pgfxy(6.5,11.5)}
  \pgflineto{\pgfxy(1,10.5)}
  \pgfcurveto{\pgfxy(0.6,9.75)}{\pgfxy(0.6,8.75)}{\pgfxy(1,8)}
  \pgflineto{\pgfxy(6.5,2)}
  \pgflineto{\pgfxy(13,3)}
  \pgfcurveto{\pgfxy(12,4)}{\pgfxy(12,5)}{\pgfxy(13,6)}
  \pgfclosepath
  
  \pgfmoveto{\pgfxy(6.5,2)}  
  \pgfcurveto{\pgfxy(6,3)}{\pgfxy(6,4)}{\pgfxy(6.5,5)}
  \pgflineto{\pgfxy(13,6)}
  \pgfstroke

  \begin{pgfscope}
    \pgfmoveto{\pgfxy(6.5,11.5)}
    \pgflineto{\pgfxy(1,10.5)}
    \pgfcurveto{\pgfxy(0.6,9.75)}{\pgfxy(0.6,8.75)}{\pgfxy(1,8)}
    \pgflineto{\pgfxy(6.5,2)}
    \pgfcurveto{\pgfxy(6,3)}{\pgfxy(6,4)}{\pgfxy(6.5,5)}
    \pgflineto{\pgfxy(13,6)}
    \pgfclosepath
    \pgfclip

    \pgfputat{\pgfxy(8.5,0)}
    {%
      \begin{pgfrotateby}{\pgfdegree{45}}
        \pgfbox[left,base]{\pgfuseshading{cover}}
      \end{pgfrotateby}
    }
  \end{pgfscope}      
  
  \begin{pgfscope}
    \pgfmoveto{\pgfxy(6.5,2)}  
    \pgfcurveto{\pgfxy(6,3)}{\pgfxy(6,4)}{\pgfxy(6.5,5)}
    \pgflineto{\pgfxy(13,6)}
    \pgfcurveto{\pgfxy(12,5)}{\pgfxy(12,4)}{\pgfxy(13,3)}
    \pgfclosepath
    \pgfclip

    \pgfputat{\pgfxy(7.5,0)}
    {%
      \begin{pgfrotateby}{\pgfdegree{30}}
        \pgfbox[left,base]{\pgfuseshading{side}}
      \end{pgfrotateby}
    }
  \end{pgfscope}      
  
  \pgfsetlinewidth{4pt}
  \color[gray]{0.2}
  \pgfmoveto{\pgfxy(6.5,11.5)}
  \pgflineto{\pgfxy(1,10.5)}
  \pgfcurveto{\pgfxy(0.6,9.75)}{\pgfxy(0.6,8.75)}{\pgfxy(1,8)}
  \pgflineto{\pgfxy(6.5,2)}
  \pgflineto{\pgfxy(13,3)}
  \pgfcurveto{\pgfxy(12,4)}{\pgfxy(12,5)}{\pgfxy(13,6)}
  \pgfclosepath
  
  \pgfmoveto{\pgfxy(6.5,2)}  
  \pgfcurveto{\pgfxy(6,3)}{\pgfxy(6,4)}{\pgfxy(6.5,5)}
  \pgflineto{\pgfxy(13,6)}
  \pgfstroke
\end{pgfpicture}

\noindent
\begin{pgfpicture}{-5.4cm}{0cm}{5.4cm}{6.5cm}
  \pgfsetlinewidth{0.8pt}
  \pgfxyline(-5,0)(5,0)
    
  \pgfsetlinewidth{0.4pt}

  \pgfheaplabeledcentered{2cm}{2.5cm}{$\Class{CFL}$}
  \pgfheaplabeledcentered{3.5cm}{3cm}{\raise10pt\hbox{}$\Class{DLINSPACE}$}
  \pgfheaplabeledcentered{5cm}{4cm}{\raise13pt\hbox{}$\Class{NLINSPACE} = \Class{CSL}$}
  \pgfheaplabeledcentered{6cm}{5cm}{$\Class{PSPACE}$}

  \pgfsetdash{{3pt}{3pt}}{0pt}
  \pgfheaplabeled{\pgfxy(0,3.3)}{\pgfxy(-5,6)}{\pgfxy(5,6)}{}%
  \pgfputat{\pgfxy(-4.6,5.75)}{\pgfbox[left,base]{$\Class{PSPACE}$-hard}}%      
\end{pgfpicture}

\def\WordBlock#1#2#3#4#5{%
  \pgfnoderect{#1}[stroke]{#2}{\pgfxy(1.5,0.5)}
  \pgfputat{#2}{%
    \pgfxyline(-0.25,-0.25)(-0.25,0.25)%
    \pgfxyline(0.25,-0.25)(0.25,0.25)%
    \pgfputat{\pgfxy(-.5,-0.1)}{\pgfbox[center,base]{#3}}
    \pgfputat{\pgfxy(0,-0.1)}{\pgfbox[center,base]{#4}}
    \pgfputat{\pgfxy(.5,-0.1)}{\pgfbox[center,base]{#5}}
  }%
}

\def\Connect#1#2{%
  \pgfmoveto{\pgfnodeborder{#1}{90}{0pt}}
  \pgflineto{\pgfnodeborder{#2}{-90}{0pt}}
  \pgfstroke}

\def\Bush{%
  \pgfsetdash{{2pt}{1pt}}{0pt}
  \pgfxyline(0,0)(-.525,0.3)
  \pgfxyline(0,0)(-.175,0.3)
  \pgfxyline(0,0)(.525,0.3)
  \pgfxyline(0,0)(.175,0.3)
  \pgfsetdash{}{0pt}}
\def\SmallBush{%
  \pgfsetdash{{2pt}{1pt}}{0pt}
  \pgfxyline(0,0)(-.29,0.29)
  \pgfxyline(0,0)(-.1,0.3)
  \pgfxyline(0,0)(.29,0.29)
  \pgfxyline(0,0)(.1,0.3)
  \pgfsetdash{}{0pt}}
\def\BoldSmallBush{%
  \pgfsetdash{{2pt}{1pt}}{0pt}
  \pgfxyline(0,0)(-.29,0.29)
  \pgfxyline(0,0)(-.1,0.3)
  \pgfxyline(0,0)(.1,0.3)
  \pgfsetdash{}{0pt}
  \Bold{\pgfxyline(0,0)(0.8,0.8)}
  \pgfputlabelrotated{0.5}{\pgfxy(0,0)}{\pgfxy(1,1)}{2pt}%
  {\pgfbox[center,base]{110}}}

\def\Label#1#2#3{%
  \pgfputlabelrotated{.5}{\pgfnodeborder{#1}{90}{0pt}}%
  {\pgfnodeborder{#2}{-90}{0pt}}{2pt}{\pgfbox[center,base]{#3}}}

\def\Rabel#1#2#3{%
  \pgfputlabelrotated{.5}{\pgfnodeborder{#2}{-90}{0pt}}%
  {\pgfnodeborder{#1}{90}{0pt}}{2pt}{\pgfbox[center,base]{#3}}}

\def\Bold#1{\pgfsetlinewidth{0.8pt}#1\pgfsetlinewidth{0.4pt}}

\begin{pgfpicture}{-5.5cm}{-0.25cm}{5.cm}{6cm}
  \Bold{\WordBlock{root}{\pgfxy(0,0)}{$\in$}{$\notin$}{$\in$}}
  \WordBlock{1}{\pgfxy(-4.5,2)}{$*$}{$*$}{$*$}
  \WordBlock{2}{\pgfxy(-1.5,2)}{$*$}{$*$}{$*$}
  \Bold{\WordBlock{3}{\pgfxy(1.5,2)}{$\notin$}{$\notin$}{$\in$}}
  \WordBlock{31}{\pgfxy(-1.5,4)}{$*$}{$*$}{$*$}
  \Bold{\WordBlock{32}{\pgfxy(0.5,4)}{$\in$}{$\in$}{$\notin$}}
  \WordBlock{33}{\pgfxy(2.5,4)}{$*$}{$*$}{$*$}
  \WordBlock{34}{\pgfxy(4.5,4)}{$*$}{$*$}{$*$}
  \WordBlock{4}{\pgfxy(4.5,2)}{$*$}{$*$}{$*$}

  \Connect{root}{1}
  \Rabel  {root}{1}{000}
  \pgfputat{\pgfnodeborder{1}{90}{0pt}}{\Bush}
  
  \Connect{root}{2}
  \Rabel  {root}{2}{001}
  \pgfputat{\pgfnodeborder{2}{90}{0pt}}{\Bush}

  \Bold{\Connect{root}{3}}
  \Label  {root}{3}{101}

  \Connect{3}{31}
  \Rabel  {3}{31}{000}
  \pgfputat{\pgfnodeborder{31}{90}{0pt}}{\SmallBush}

  \Bold{\Connect{3}{32}}
  \Rabel  {3}{32}{001}
  \pgfputat{\pgfnodeborder{32}{90}{0pt}}{\BoldSmallBush}

  \Connect{3}{33}
  \Label  {3}{33}{101}
  \pgfputat{\pgfnodeborder{33}{90}{0pt}}{\SmallBush}

  \Connect{3}{34}
  \Label  {3}{34}{110}
  \pgfputat{\pgfnodeborder{34}{90}{0pt}}{\SmallBush}
  
  \Connect{root}{4}
  \Label  {root}{4}{110}
  \pgfputat{\pgfnodeborder{4}{90}{0pt}}{\Bush}
\end{pgfpicture}

\newcommand{\Node}[3]{%
  \pgfnodecircle{#1}[stroke]{#2}{0.3cm}%
  \pgfputat{\pgfrelative{#2}{\pgfxy(0,-.075)}}{\pgfbox[center,base]{#3}}}

\newcommand{\Claim}[2]{%
  \pgfputat{\pgfrelative{\pgfxy(0.4,-0.075)}{\pgfnodecenter{#1}}}%
  {\pgfbox[left,base]{#2}}}

\renewcommand{\Bush}[3]{%
  \pgfnodecircle{#1}[virtual]{\pgfrelative{\pgfnodecenter{#2}}{#3}}{1pt}%
  \pgfnodeconnline{#2}{#1}}

\begin{pgfpicture}{-2.3cm}{-1.5cm}{1.8cm}{5.25cm}

  \Node{A}{\pgfxy(0.5,0)}{$w_1$}
  \Node{B}{\pgfxy(-1,1.5)}{$w_2$}
  \Node{C}{\pgfxy(-2,3)}{$w_3$}
  \Node{D}{\pgfxy(0,3)}{$w_4$}
  \Node{E}{\pgfxy(-0.75,4.5)}{$w_5$}

  \pgfnodeconnline{A}{B}
  \pgfnodeconnline{B}{C}
  \pgfnodeconnline{B}{D}
  \pgfnodeconnline{D}{E}
  
  \pgfnodelabelrotated{B}{A}[.5][2pt]{\pgfbox[center,base]{0}}
  \pgfnodelabelrotated{C}{B}[.5][2pt]{\pgfbox[center,base]{0}}
  \pgfnodelabelrotated{B}{D}[.5][2pt]{\pgfbox[center,base]{1}}
  \pgfnodelabelrotated{E}{D}[.5][2pt]{\pgfbox[center,base]{0}}
  
  \Claim{A}{`$\in A$'}
  \Claim{B}{`$\in A$'}
  \Claim{C}{`$\in B$'}
  \Claim{D}{`$\in A$'}

  \Bush{In}{A}{\pgfxy(-1,-.75)}
  \pgfsetdash{{2pt}{1pt}}{0pt}
  \Bush{Ar}{A}{\pgfxy(.75,.75)}
  \Bush{Cl}{C}{\pgfxy(-.375,.75)}
  \Bush{Cr}{C}{\pgfxy(.375,.75)}
  \Bush{Dr}{D}{\pgfxy(.375,.75)}
  \Bush{El}{E}{\pgfxy(-.3,.75)}
  \Bush{Er}{E}{\pgfxy(.3,.75)}
\end{pgfpicture}
\begin{pgfpicture}{-2.8cm}{-1.5cm}{3.7cm}{5.25cm}

  \Node{A}{\pgfxy(0,0)}{$w_1$}
  \Node{B}{\pgfxy(-1.5,1.5)}{$w_3$}
  \Node{C}{\pgfxy(-2.5,3)}{$w_2$}
  \Node{D}{\pgfxy(1.5,1.5)}{$w_4$}
  \Node{E}{\pgfxy(0.5,3)}{$w_5$}

  \pgfnodeconnline{A}{B}
  \pgfnodeconnline{B}{C}
  \pgfnodeconnline{A}{D}
  \pgfnodeconnline{D}{E}
  
  \pgfnodelabelrotated{B}{A}[.5][2pt]{\pgfbox[center,base]{0}}
  \pgfnodelabelrotated{A}{D}[.5][2pt]{\pgfbox[center,base]{1}}
  \pgfnodelabelrotated{C}{B}[.5][2pt]{\pgfbox[center,base]{0}}
  \pgfnodelabelrotated{E}{D}[.5][2pt]{\pgfbox[center,base]{0}}

  \Claim{A}{`${\in}\kern2pt A$'}
  \Claim{B}{`${\in}\kern2pt A$', `$\in B$'}
  \Claim{D}{`$\in A$', `$\in B$'}
  
  \Bush{In}{A}{\pgfxy(1,-.75)}
  \pgfsetdash{{2pt}{1pt}}{0pt}
  \Bush{Br}{B}{\pgfxy(.5,.75)}
  \Bush{Dr}{D}{\pgfxy(.5,.75)}
  \Bush{Cl}{C}{\pgfxy(-.375,.75)}
  \Bush{Cr}{C}{\pgfxy(.375,.75)}
  \Bush{El}{E}{\pgfxy(-.375,.75)}
  \Bush{Er}{E}{\pgfxy(.375,.75)}
\end{pgfpicture}

\begin{pgfpicture}{-5.4cm}{-3.2cm}{4.25cm}{3.2cm}
  \begin{pgfautomaton}
    \pgfsetstatemooreradius{1.05cm}
    
    \pgfstatemoore{eq}{\pgfxy(-3,0)}
    {$q_=$}{$\{00,11\}$}
    \pgfstatemoore{neq}{\pgfxy(2,0)}%
    {$q_{\not\sqsubseteq,\not\sqsupseteq}$}{$\{00,01,10\}$}
    \pgfstatemoore{less}{\pgfxy(2,2.3)}
    {$q_{\sqsubseteq}$}{$\{00,10,11\}$} 
    \pgfstatemoore{more}{\pgfxy(2,-2.3)}
    {$q_{\sqsupseteq}$}{$\{00,01,11\}$} 

    \pgfstateinitial{eq}[left]{start}

    \pgfstateconnectrotated{eq}{less}[.6]{$\binom{\square}{x}$}
    \pgfstateconnectrotated{eq}{neq}[.6]{$\binom{0}{1},\binom{1}{0}$}
    \pgfstateconnectrotated{eq}{more}[.6]{$\binom{x}{\square}$}

    \pgfstateloop{eq}{\pgfdirection{below}}{$\binom{x}{x}$}
    \pgfstateloop{less}{\pgfdirection{right}}{$\binom{y}{z}$}
    \pgfstateloop{neq}{\pgfdirection{right}}{$\binom{y}{z}$}
    \pgfstateloop{more}{\pgfdirection{right}}{$\binom{y}{z}$}
  \end{pgfautomaton}
\end{pgfpicture}

\begin{pgfpicture}{-5.1cm}{-2.1cm}{5cm}{4.3cm}
  \begin{pgfautomaton}
    \pgfstatemoore{a}{\pgfxy(-3,0)}{$q_1$}{$\{0,1\}$}
    \pgfstatemoore{b}{\pgfxy(0,1.2)}{$q_2$}{$\{0,1\}$}
    \pgfstatemoore{c}{\pgfxy(3,1.2)}{$q_3$}{$\{0,2\}$}
    \pgfstatemoore{d}{\pgfxy(0,-1.2)}{$q_4$}{$\{0,1\}$}

    \pgfstateinitial{a}[left]{start}

    \pgfstateconnectrotated{a}{b}[.6]{$\binom{1}{1}$}
    \pgfstateconnectrotated{b}{c}[.6]{$\binom{x}{y}$}
    \pgfstateconnectrotated{a}{d}[.6]{$\binom{x}{y}$}

    \pgfstateloop{a}{\pgfdirection{below}}{$\binom{0}{0}$}
    \pgfstateloop{b}{\pgfdirection{above}}{$\binom{z}{z}$}
    \pgfstateloop{c}{\pgfdirection{right}}{$\binom{u}{v}$}
    \pgfstateloop{d}{\pgfdirection{right}}{$\binom{u}{v}$}
  \end{pgfautomaton}
\end{pgfpicture}

\def\QuadA#1#2#3{
  \pgfmoveto{\pgfxyz(#1,#2,#2)}
  \pgflineto{\pgfxyz(#1,#2,#3)}
  \pgflineto{\pgfxyz(#1,#3,#3)}
  \pgflineto{\pgfxyz(#1,#3,#2)}
  \pgfclosepath
  \pgfstroke}
\def\QuadB#1#2#3{
  \pgfmoveto{\pgfxyz(#2,#1,#2)}
  \pgflineto{\pgfxyz(#2,#1,#3)}
  \pgflineto{\pgfxyz(#3,#1,#3)}
  \pgflineto{\pgfxyz(#3,#1,#2)}
  \pgfclosepath
  \pgfstroke}
\def\QuadC#1#2#3{
  \pgfmoveto{\pgfxyz(#2,#2,#1)}
  \pgflineto{\pgfxyz(#2,#3,#1)}
  \pgflineto{\pgfxyz(#3,#3,#1)}
  \pgflineto{\pgfxyz(#3,#2,#1)}
  \pgfclosepath
  \pgfstroke}
\def\HyperCube{
  \QuadA003 \QuadA303 \QuadA112
  \QuadA212 \QuadB003 \QuadB303
  \QuadB112 \QuadB212 \QuadC003
  \QuadC303 \QuadC112 \QuadC212
  \pgfline{\pgfxyz(0,0,0)}{\pgfxyz(1,1,1)}
  \pgfline{\pgfxyz(3,0,0)}{\pgfxyz(2,1,1)}
  \pgfline{\pgfxyz(0,3,0)}{\pgfxyz(1,2,1)}
  \pgfline{\pgfxyz(0,0,3)}{\pgfxyz(1,1,2)}
  \pgfline{\pgfxyz(3,3,0)}{\pgfxyz(2,2,1)}
  \pgfline{\pgfxyz(0,3,3)}{\pgfxyz(1,2,2)}
  \pgfline{\pgfxyz(3,0,3)}{\pgfxyz(2,1,2)}
  \pgfline{\pgfxyz(3,3,3)}{\pgfxyz(2,2,2)}}

\makeatletter
\def\Shear#1{\pgf@process{#1}\divide\pgf@y by 3\advance\pgf@x by .6\pgf@y}
\makeatother
  
\begin{pgfpicture}{0cm}{0cm}{10cm}{2cm}
  \pgfsetroundjoin
  \pgfsetxvec{\Shear{\pgfpolar{0}{.25cm}}}
  \pgfsetyvec{\pgfpolar{90}{.25cm}}
  \pgfsetzvec{\Shear{\pgfpolar{-90}{0.25cm}}}
  \HyperCube

  \pgftranslateto{\pgfpoint{1.5cm}{0cm}}
  \pgfsetxvec{\Shear{\pgfpolar{-10}{.25cm}}}
  \pgfsetyvec{\pgfpolar{90}{.25cm}}
  \pgfsetzvec{\Shear{\pgfpolar{-100}{0.25cm}}}
  \HyperCube

  \pgftranslateto{\pgfpoint{1.5cm}{0cm}}
  \pgfsetxvec{\Shear{\pgfpolar{-20}{.25cm}}}
  \pgfsetyvec{\pgfpolar{90}{.25cm}}
  \pgfsetzvec{\Shear{\pgfpolar{-110}{0.25cm}}}
  \HyperCube

  \pgftranslateto{\pgfpoint{1.5cm}{0cm}}
  \pgfsetxvec{\Shear{\pgfpolar{-30}{.25cm}}}
  \pgfsetyvec{\pgfpolar{90}{.25cm}}
  \pgfsetzvec{\Shear{\pgfpolar{-120}{0.25cm}}}
  \HyperCube

  \pgftranslateto{\pgfpoint{1.5cm}{0cm}}
  \pgfsetxvec{\Shear{\pgfpolar{-40}{.25cm}}}
  \pgfsetyvec{\pgfpolar{90}{.25cm}}
  \pgfsetzvec{\Shear{\pgfpolar{-130}{0.25cm}}}
  \HyperCube

  \pgftranslateto{\pgfpoint{1.5cm}{0cm}}
  \pgfsetxvec{\Shear{\pgfpolar{-50}{.25cm}}}
  \pgfsetyvec{\pgfpolar{90}{.25cm}}
  \pgfsetzvec{\Shear{\pgfpolar{-140}{0.25cm}}}
  \HyperCube

  \pgftranslateto{\pgfpoint{1.5cm}{0cm}}
  \pgfsetxvec{\Shear{\pgfpolar{-60}{.25cm}}}
  \pgfsetyvec{\pgfpolar{90}{.25cm}}
  \pgfsetzvec{\Shear{\pgfpolar{-150}{0.25cm}}}
  \HyperCube

  \pgftranslateto{\pgfpoint{1.5cm}{0cm}}
  \pgfsetxvec{\Shear{\pgfpolar{-70}{.25cm}}}
  \pgfsetyvec{\pgfpolar{90}{.25cm}}
  \pgfsetzvec{\Shear{\pgfpolar{-160}{0.25cm}}}
  \HyperCube

  \pgftranslateto{\pgfpoint{1.5cm}{0cm}}
  \pgfsetxvec{\Shear{\pgfpolar{-80}{.25cm}}}
  \pgfsetyvec{\pgfpolar{90}{.25cm}}
  \pgfsetzvec{\Shear{\pgfpolar{-170}{0.25cm}}}
  \HyperCube

  \pgftranslateto{\pgfpoint{1.5cm}{0cm}}
  \pgfsetxvec{\Shear{\pgfpolar{-90}{.25cm}}}
  \pgfsetyvec{\pgfpolar{90}{.25cm}}
  \pgfsetzvec{\Shear{\pgfpolar{-180}{0.25cm}}}
  \HyperCube
\end{pgfpicture}



\section{Basic Graphic Drawing}


\subsection{Main Environments}

In order to draw a picture using \pgf, you have to put the picture
inside the environment |pgfpicture| or the environment
|pgfpictureboxed|.

\begin{environment}{{pgfpicture}\marg{lower left x}\marg{lower
      left y}\marg{upper right x}\marg{upper right y}} 
  Sets up a \pgf-picture environment. The dimensions tell \TeX\ how much
  space it should reserve. These sizes are not used for clipping.

  Inside this environment, \TeX's null font is used. This causes any
  text (including spaces) that is not put inside a |\pgfbox| to be
  suppressed. Furthermore, the dimensions 1ex and 1em are zero. To
  access the original dimensions, you can use the dimensions
  \declare{\texttt{\string\pgfex}} and \declare{\texttt{\string\pgfem}}, which are setup are the
  beginning of the environment to the value of 1ex and 1em,
  respectively. 
  \example
\begin{verbatim}
\begin{pgfpicture}{0cm}{0cm}{1cm}{1cm}
  \pgfline{\pgforigin}{\pgfpoint{10pt}{10pt}}
\end{pgfpicture}
\end{verbatim}
\end{environment}

\begin{environment}{{pgfpictureboxed}\marg{lower left x}\marg{lower
      left y}\marg{upper right x}\marg{upper right y}}
  Identical to |pgfpicture|, except that a frame of the size of
  the picture is drawn around it.
\end{environment}

Inside a picture, you can create nested scopes using
|pgfscope|. Changes made inside a |pgfscope| are undone when the scope
ends.

\begin{environment}{{pgfscope}}
  All changes made inside a scope are local to that scope.
  \example

\begin{pgfpicture}{0cm}{0cm}{5cm}{0.75cm}
  \pgfxyline(0,0)(5,0)
  \begin{pgfscope}
    \pgfsetlinewidth{2pt}
    \pgfxyline(0,0.25)(5,0.25)
  \end{pgfscope}
  \pgfxyline(0,0.5)(5,0.5)
\end{pgfpicture}
\begin{verbatim}
\begin{pgfpicture}{0cm}{0cm}{5cm}{0.75cm}
  \pgfxyline(0,0)(5,0)
  \begin{pgfscope}
    \pgfsetlinewidth{2pt}
    \pgfxyline(0,0.25)(5,0.25)
  \end{pgfscope}
  \pgfxyline(0,0.5)(5,0.5)
\end{pgfpicture}
\end{verbatim}
\end{environment}



\subsection{How to Specify a Point}

\pgf\ uses a two dimensional coordinate system that is local to the
current picture been drawn. A point inside the coordinate system can
be specified using the command |pgfpoint|. You can use all
dimensions available in \TeX\ when specifying a dimension.

\begin{command}{\pgforigin}
  Yields the origin.
  \example |\pgmoveto{\pgforigin}|
\end{command}


\begin{command}{\pgfpoint\marg{x coordinate}\marg{y coordinate}}
  Yields a point location. The coordinates are given as \TeX\ dimensions.
  \example |\pgfline{\pgfpoint{10sp}{-1.5cm}}{\pgfpoint{10pt}{1cm}}|
\end{command}

\begin{command}{\pgfpolar\marg{degree}\marg{radius}}
  Yields a point location given in polar coordinates.
  \example |\pgfmoveto{\pgfpolar{30}{1cm}}|
\end{command}

\begin{command}{\pgfdirection\marg{direction string}}
    Returns the degree that corresponds to the direction.
  Allowed values for \meta{direction string} are
  |n[orth]|, |s[south]|, |e[east]|,
  |w[est]|, |ne|, |nw|, |se|, and |sw|.
  \example |\pgfmoveto{\pgfpolar{\pgfdirection{n}}{1cm}}|
\end{command}

Coordinates can also be specified as multiples of an $x$-vector and a
$y$-vector. Normally, the $x$-vector points one centimeter in the
$x$-direction and the $y$-vector points one centimeter in the
$y$-direction, but using the commands |pgfsetxvec| and
|pgfsetyvec| they can be changed.

It is also possible to specify a point as a multiple of three vectors,
the $x$-, $y$-, and $z$-vector. This is useful for creating simple
three dimensional graphics.

\begin{command}{\pgfxy|(|\meta{$s_x$}|,|\meta{$s_y$}|)|}
  Yields a point that is situated at $s_x$ times the
  $x$-vector plus $s_y$ times the $y$-vector.
  \example |\pgfline{\pgfxy(0,0)}{\pgfxy(1,1)}|
\end{command}

\begin{command}{\pgfxyz|(|\meta{$s_x$}|,|\meta{$s_y$}|,|\meta{$s_z$}|)|}
  Yields a point that is situated at $s_x$ times the
  $x$-vector plus $s_y$ times the $y$-vector plus  $s_z$ times the
  $z$-vector.
  \example

\begin{pgfpicture}{-0.5cm}{-0.5cm}{1cm}{1.25cm}
  \pgfsetendarrow{\pgfarrowto}
  \pgfline{\pgfxyz(0,0,0)}{\pgfxyz(0,0,1)}
  \pgfline{\pgfxyz(0,0,0)}{\pgfxyz(0,1,0)}
  \pgfline{\pgfxyz(0,0,0)}{\pgfxyz(1,0,0)}
\end{pgfpicture}
\begin{verbatim}
  \pgfsetendarrow{\pgfarrowto}
  \pgfline{\pgfxyz(0,0,0)}{\pgfxyz(0,0,1)}
  \pgfline{\pgfxyz(0,0,0)}{\pgfxyz(0,1,0)}
  \pgfline{\pgfxyz(0,0,0)}{\pgfxyz(1,0,0)}
\end{verbatim}
\end{command}

\begin{command}{\pgfsetxvec\marg{point}}
  A point that replaces the current $x$-vector. The commands
  \declare{\texttt{\string\pgfsetyvec}} and
  \declare{\texttt{\string\pgfsetzvec}} are defined the same way.
  \example
\begin{verbatim}
  \pgfsetxvec{\pgfpoint{2cm}{0cm}}
  \pgfline{\pgfxy(0,0)}{\pgfxy(1,1)}
  % Same as \pgfline{\pgforigin}{\pgfpoint{2cm}{1cm}}
\end{verbatim}
\end{command}



There exist different commands for treating points as vectors.

\begin{command}{\pgfdiff\marg{point $p_1$}\marg{point $p_2$}}
  Yields the difference vector $p_2 - p_1$.
  \example |\pgfmoveto{\pgfdiff{\pgfxy(1,1)}{\pgfxy(2,3)}}|
\end{command}


\begin{command}{\pgfrelative\marg{point $p_1$}\marg{point $p_2$}}
  Yields the  the sum $p_1 + p_2$
  \example  |\pgfmoveto{\pgfrelative{\pgfxy(0,1)}{\pgfpoint{1pt}{2pt}}}|
\end{command}


\begin{command}{\pgfpartway\marg{scalar $r$}\marg{point $p_1$}\marg{point $p_2$}}
  Yields a point that is the $r$th fraction between $p_1$
  and~$p2$, that is, $p_1 + r(p_2-p_1)$. For $r=0.5$ the middle
  between $p_1$ and~$p_2$ is returned.
  \example |\pgfmoveto{\pgfpartway{0.5}{\pgfxy(1,1)}{\pgfxy(2,3)}}|
\end{command}

\begin{command}{\pgfbackoff\marg{distance}\marg{start point}\marg{end point}}
  Yields a point that is located \meta{distance} many units removed
  from the start point in the direction of the end point.
  \example
\begin{verbatim}
\pgfline{\pgfbackoff{2pt}{\pgfxy(1,1)}{\pgfxy(2,3)}}
        {\pgfbackoff{3pt}{\pgfxy(2,3)}{\pgfxy(1,1)}}
\end{verbatim}
\end{command}



\subsection{Coordinate Systems}

Coordinate systems can be translated, rotated, and magnified using two
environments. \emph{Please note that these operations are incompatible
  with the node drawing commands.} Note also that the magnify
operation also makes lines appear bigger. If this is not desired, you
might wish to enlarge the $x$- and $y$-vectors instead.

\begin{environment}{{pgftranslate}\marg{new origin}}
  Makes \meta{new origin} the new origin within the scope of
  the environment.
  \example
\begin{verbatim}
\begin{pgftranslate}{\pgfpoint{0cm}{1cm}}
  \pgfline{\pgforigin}{\pgfxy(1,0)}
\end{pgftranslate}
\end{verbatim}
\end{environment}

\begin{command}{\pgftranslateto\marg{new origin}}
  Makes the parameter the new origin.
  \example
\begin{verbatim}
\pgftranslateto{\pgfpoint{0cm}{1cm}}
\pgfline{\pgforigin}{\pgfxy(1,0)}
\end{verbatim}
\end{command}

\begin{command}{\pgfputat\marg{an origin}\marg{commands}}
  Executes the commands after having translated the origin
  to the given location.
  \example |\pgfputat{\pgfxy(1,0)}{\pgfbox[center,center]{Hello world}}|
\end{command}


\begin{environment}{{pgfrotateby}\marg{sin/cos of rotation degree}}
  Rotates the current coordinate system by \marg{sin/cos of rotation
    degree} within the scope of the environment. Use |\pgfdegree| to
  calculate the rotation degree.  
  \example
\begin{verbatim}
\begin{pgfrotateby}{\pgfdegree{30}}
  \pgfline{\pgforigin}{\pgfxy(1,0)}
\end{pgfrotateby}
\end{verbatim}
\end{environment}

\begin{environment}{{pgfmagnify}\marg{x magnification}\marg{y magnification}}
  Magnifies everything within the environment by the given
  factors.
  \example
\begin{verbatim}
\begin{pgfmagnify}{2}{2}
  \pgfline{\pgforigin}{\pgfxy(1,0)}
\end{pgfmagnify}
\end{verbatim}
\end{environment}




\subsection{Path Construction}

Lines and shapes can be drawn by constructing paths and by then
stroking and filling them. In order to construct a path, you must
first use the command |\pgfmoveto|, followed by a series of
|\pgflineto| and |\pgfcurveto| commands. You can use
|\pgfclosepath| to create a closed shape. You can also use
|\pgfmoveto| commands while constructing a path.


\begin{command}{\pgfmoveto\marg{point}}
  Makes \meta{point} the current point.
  \example |\pgfmoveto{\pgforigin}|
\end{command}


\begin{command}{\pgflineto\marg{point}}
  Extends the path by a straight line from the current point to
  \meta{point}. This point is then made the current point.
  \example
\begin{verbatim}
  \pgfmoveto{\pgforigin}
  \pgflineto{\pgfxy(1,1)}
\end{verbatim}
\end{command}


\begin{command}{\pgfcurveto\marg{support point 1}\marg{support point 2}\marg{point}}
  Extends the path by a curve from the current point to
  \meta{point}. This point is then made the current point. The support
  points govern in which direction the curves head 
  at the start and at the end. At the start it will head in a
  straight line towards \marg{support point 1}, at the destination it
  will head in a straight line towards the destination as if it came
  from \marg{support point 2}.
  \example

\begin{pgfpicture}{0cm}{0cm}{2cm}{1.25cm}
  \pgfmoveto{\pgforigin}
  \pgfcurveto{\pgfxy(1,1)}{\pgfxy(2,1)}{\pgfxy(2,0)}
  \pgfstroke

  \pgfcircle[fill]{\pgforigin}{2pt}
  \pgfcircle[fill]{\pgfxy(1,1)}{2pt}
  \pgfcircle[fill]{\pgfxy(2,1)}{2pt}
  \pgfcircle[fill]{\pgfxy(2,0)}{2pt}

  \pgfputat{\pgfxy(1.1,1)}{\pgfbox[left,center]{\footnotesize $(1,1)$}}
  \pgfputat{\pgfxy(2.1,1)}{\pgfbox[left,center]{\footnotesize $(2,1)$}}
  \pgfputat{\pgfxy(2.1,0)}{\pgfbox[left,center]{\footnotesize $(2,0)$}}
\end{pgfpicture}
\begin{verbatim}
  \pgfmoveto{\pgforigin}
  \pgfcurveto{\pgfxy(1,1)}{\pgfxy(2,1)}{\pgfxy(2,0)}
  \pgfstroke
\end{verbatim}
\end{command}

\begin{command}{\pgfclosepath}
  Connects the current point to the point where the current path
  started.
\end{command}


\begin{command}{\pgfzerocircle\marg{radius}}
  Appends a circle around the origin of the given radius to the
  current path.
  \example |\pgfzerocircle{1cm}|
\end{command}

\begin{command}{\pgfzeroellipse\marg{axis vector 1}\marg{axis vector 2}}
  Appends an ellipse with the given axis vectors centered at the
  origin to the current path.
  \example
\begin{pgfpicture}{-1cm}{-1cm}{1cm}{1cm}
  \pgfzeroellipse{\pgfxy(0.5,0.5)}{\pgfxy(-0.75,0.75)}
  \pgfstroke
  \pgfline{\pgforigin}{\pgfxy(0.5,0.5)}
  \pgfline{\pgforigin}{\pgfxy(-0.75,0.75)}
\end{pgfpicture}
\begin{verbatim}
  \pgfzeroellipse{\pgfxy(0.5,0.5)}{\pgfxy(-0.75,0.75)}
  \pgfstroke
  \pgfline{\pgforigin}{\pgfxy(0.5,0.5)}
  \pgfline{\pgforigin}{\pgfxy(-0.75,0.75)}
\end{verbatim}
\end{command}

The basic drawing commands also come in `quick' versions. These
versions get plain numbers as input that represent \TeX\ points. These
commands are executed much quicker than the normal commands. They are
useful if you need to do construct very long or numerous paths.

\begin{command}{\pgfqmoveto\marg{x bp}\marg{y bp}}
  Makes the given point the current point. The real numbers given are
  interpreted as \TeX\ ``big points,'' which are a 1/72th of an inch
  (as opposed to \TeX\ points, which are a 1/72.27th of an inch).
  \example |\pgfqmoveto{10}{20}|
\end{command}

\begin{command}{\pgfqlineto\marg{x bp}\marg{y bp}}
  Extends the path by a straight line from the current point to the
  parameter point. The parameter point is then made the current
  point.
  \example
\begin{verbatim}
  \pgfqmoveto{0}{0}
  \pgfqlineto{100}{100}
  \pgfstroke
\end{verbatim}
\end{command}

\begin{command}{\pgfqcurveto\marg{$s^1_x$ bp}\marg{$s^1_y$ bp}\marg{$s^2_x$ bp}\marg{$s^2_y$ bp}\marg{x bp}\marg{y bp}}
  Quick version of the |\pgfcurveto| command.
  \example
\begin{verbatim}
  \pgfqmoveto{0}{0}
  \pgfqcurveto{100}{100}{200}{100}{200}{0}
  \pgfstroke
\end{verbatim}
\end{command}



\subsection{Stroking and Filling}

Once you have constructed a path, you can use the commands
|\pgfstroke| and |\pgffill| to paint the path. How the
path is painted depends on a number of parameters: For filling, the
fill color is important (the fill color is the same as the stroke
color and it set by using the standard |\color| commands from the
|color| package or any compatible package). For stroking, the
line width, the line dashing, the miter join, and the cap form are
furthermore of importance.

\begin{command}{\pgfstroke}
  Draws the current path with current color, thickness, dashing,
  miter, and cap. If an arrow type is set up, arrows are drawn at the
  beginning and at the end.
  \example
\begin{verbatim}
  \pgfmoveto{\pgforigin}
  \pgflineto{\pgfxy(1,1)}
  \pgfstroke
\end{verbatim}
\end{command}


\begin{command}{\pgfqstroke}
  Like |\pgfstroke|, except that no arrows are drawn.
\end{command}


\begin{command}{\pgfclosestroke}
  Closes the current path and then draws it.
  \example
\begin{verbatim}
  \pgfmoveto{\pgforigin}
  \pgflineto{\pgfxy(1,1)}
  \pgflineto{\pgfxy(0,1)}
  \pgfclosestroke
\end{verbatim}
\end{command}

\begin{command}{\pgffill}
  Closes the current path, if necessary, and then fill the area with
  the current color.
  \example
\begin{verbatim}
  \pgfmoveto{\pgforigin}
  \pgflineto{\pgfxy(1,1)}
  \pgfstroke
\end{verbatim}
\end{command}

\begin{command}{\pgfeofill}
  Same as |\pgffill|, except that the even-odd rule is used.
\end{command}

\begin{command}{\pgffillstroke}
  Strokes the current path, the closes the current path, if necessary,
  and then fills the area with the current color.
\end{command}

\begin{command}{\pgfeofillstroke}
  Same as |\pgffillstroke|, except that the even-odd rule is used.
\end{command}


\begin{command}{\pgfsetlinewidth\marg{line width}}
  Sets the line width for subsequent stroking commands to \meta{line
    width}. A dimension of 0pt corresponds to the thinnest drawable 
  line. On high resolution printers these will be impossible to see.
  \example |\pgfsetlinewidth{3pt}|
\end{command}



\begin{command}{\pgfsetdash\marg{list of even length of dimensions}\marg{phase}}
  Sets the dashing of a line. The first entry in the list specifies
  the length of the first solid part of the list. The second entry
  specifies the length of the following gap. Then comes the length of
  the second solid part, following by the length of the second gap,
  and so on. The \meta{phase} specifies where the first solid part
  starts relative to the beginning of the line.
  \example

\begin{pgfpicture}{0cm}{0.5cm}{5cm}{1.25cm}
  \pgfsetdash{{0.5cm}{0.5cm}{0.1cm}{0.2cm}}{0cm}
  \pgfxyline(0,1)(5,1)
  \pgfsetdash{{0.5cm}{0.5cm}{0.1cm}{0.2cm}}{0.1cm}
  \pgfxyline(0,0.9)(5,0.9)
  \pgfsetdash{{0.5cm}{0.5cm}{0.1cm}{0.2cm}}{0.2cm}
  \pgfxyline(0,0.8)(5,0.8)
\end{pgfpicture}
\begin{verbatim}
  \pgfsetdash{{0.5cm}{0.5cm}{0.1cm}{0.2cm}}{0cm}
  \pgfxyline(0,1)(5,1)
  \pgfsetdash{{0.5cm}{0.5cm}{0.1cm}{0.2cm}}{0.1cm}
  \pgfxyline(0,0.9)(5,0.9)
  \pgfsetdash{{0.5cm}{0.5cm}{0.1cm}{0.2cm}}{0.2cm}
  \pgfxyline(0,0.8)(5,0.8)
\end{verbatim}
\end{command}


\begin{command}{\pgfsetbuttcap}
  Set a butt line cap for subsequent stroking commands.
\end{command}

\begin{command}{\pgfsetroundcap}
  Set a round line cap for subsequent stroking commands.
\end{command}

\begin{command}{\pgfsetrectcap}
  Set a rectangular line cap for subsequent stroking commands.
\end{command}

\begin{command}{\pgfsetbeveljoin}
  Set a bevel line join for subsequent stroking commands.
\end{command}

\begin{command}{\pgfsetroundjoin}
  Set a round line join for subsequent stroking commands.
\end{command}

\begin{command}{\pgfsetmiterjoin}
  Set a miter line join for subsequent stroking commands.
\end{command}


\begin{command}{\pgfsetmiterlimit\marg{miter limit}}
  Sets the miter limit for subsequent stroking commands. See the
  \textsc{pdf} manual for details on what the miter limit is.
  \example |\pgfsetmiterlimit{3pt}|
\end{command}



\subsection{Clipping}

Paths can also be used to clip subsequent drawings. Executing the clip
operator intersects the current clipping area with the area specified
by the path. There is no way of enlarging the clipping area once
more. However, if a clipping operations is done inside a
\text{pgfscope} environment, the end of the scope restores the
original clipping area.

\begin{command}{\pgfclip}
  Closes the current path and intersect it with the current clipping
  path to form a new clipping path.
  \example

\begin{pgfpicture}{0cm}{0cm}{2cm}{1.25cm}
  \pgfmoveto{\pgfxy(0,0)}
  \pgflineto{\pgfxy(0,1)}
  \pgflineto{\pgfxy(1,0)}
  \pgfclip

  \pgfcircle[fill]{\pgfxy(0.25,0.25)}{14pt}
\end{pgfpicture}
\begin{verbatim}
  \pgfmoveto{\pgfxy(0,0)}
  \pgflineto{\pgfxy(0,1)}
  \pgflineto{\pgfxy(1,0)}
  \pgfclip

  \pgfcircle[fill]{\pgfxy(0.25,0.25)}{14pt}
\end{verbatim}
\end{command}


\begin{command}{\pgfstrokeclip}
  Stroke the current path, then close it, and intersect it with the
  current clipping path to form a new clipping path.
\example

\begin{pgfpicture}{0cm}{0cm}{2cm}{1.25cm}
  \pgfmoveto{\pgfxy(0,0)}
  \pgflineto{\pgfxy(0,1)}
  \pgflineto{\pgfxy(1,0)}
  \pgfstrokeclip

  \pgfcircle[fill]{\pgfxy(0.25,0.25)}{14pt}
\end{pgfpicture}
\begin{verbatim}
  \pgfmoveto{\pgfxy(0,0)}
  \pgflineto{\pgfxy(0,1)}
  \pgflineto{\pgfxy(1,0)}
  \pgfstrokeclip

  \pgfcircle[fill]{\pgfxy(0.25,0.25)}{14pt}
\end{verbatim}
\end{command}


\begin{command}{\pgfclosestrokeclip}
  Close the current path, strokes it, and intersect it with the
  current clipping path to form a new clipping path.
  \example

\begin{pgfpicture}{0cm}{0cm}{2cm}{1.25cm}
  \pgfmoveto{\pgfxy(0,0)}
  \pgflineto{\pgfxy(0,1)}
  \pgflineto{\pgfxy(1,0)}
  \pgfclosestrokeclip

  \pgfcircle[fill]{\pgfxy(0.25,0.25)}{14pt}
\end{pgfpicture}
\begin{verbatim}
  \pgfmoveto{\pgfxy(0,0)}
  \pgflineto{\pgfxy(0,1)}
  \pgflineto{\pgfxy(1,0)}
  \pgfclosestrokeclip

  \pgfcircle[fill]{\pgfxy(0.25,0.25)}{14pt}
\end{verbatim}
\end{command}


\begin{command}{\pgffillclip}
  Closes the current path, fills it, and intersect it with the
  current clipping path to form a new clipping path.
\end{command}


\begin{command}{\pgffillstrokeclip}
  Closes the current path, fills it, strokes it, and intersect it with the
  current clipping path to form a new clipping path.
\end{command}




\subsection{Shape and Line Drawing}

There are several commands that make drawing shapes and lines
easier. However, in principle these could be implemented using the
path construction and stroking and filling commands introduced above.

\begin{command}{\pgfline\marg{start point}\marg{end point}}
  Draws a line from \meta{start point} to \meta{end point}. This command is
  equivalent to constructing a path from the start to the end point
  and then stroking it.
  \example |\pgfline{\pgfxy(0,0)}{\pgfxy(1,1)}|
\end{command}

\begin{command}{\pgfxyline|(|\meta{$x_1$}|,|\meta{$y_1$}|),(|\meta{$x_2$}|,|\meta{$y_2$}|)|}
  Like the |\pgfline| command, except the start and end points
  are given in $xy$-coordinates.
  \example |\pgfxyline(0,0)(1,1)|
\end{command}

\begin{command}{\pgfcurve\marg{start point}\marg{support point
      1}\marg{support point 2}\marg{end point}}
  Draws a curve from the start to the end point with given support
  points.
  \example |\pgfcurve{\pgfxy(0,0)}{\pgfxy(0,1)}{\pgfxy(1,1)}{\pgfxy(1,0)}| 
\end{command}

\begin{command}{\pgfxycurve|(|\meta{$x_1$}|,|\meta{$y_1$}|),(|\meta{$x_1'$}|,|\meta{$y_1'$}|),(|\meta{$x_2'$}|,|\meta{$y_2'$}|),(|\meta{$x_2$}|,|\meta{$y_2$}|)|}
  Like the |\pgfcurve| command, except that all points 
  are given in $xy$-coordinates.
  \example |\pgfxycurve(0,0)(0,1)(1,1)(1,0)|
\end{command}



\begin{command}{\pgfrect\ooarg{drawing type}\marg{lower left
      corner}\marg{height/width vector}}
  Draws a rectangle. The \meta{drawing type} can be |stroke|, |fill|,
  |fillstroke|, or |clip|. 
  \example
\begin{verbatim}
  % Draw a filled rectangle with corners (2,2) and (3,3)
  \pgfrect[fill]{\pgfxy(2,2)}{\pgfxy(1,1)}
\end{verbatim}
\end{command}


\begin{command}{\pgfcircle\ooarg{drawing type}\marg{center}\marg{radius}}
  Draws a circle centered at \meta{center} of radius
  \meta{radius}. The \meta{drawing type} can be   |stroke|, |fill|, or
  |fillstroke|.
  \example |\pgfcircle[stroke]{\pgfxy(1,1)}{10pt}|
\end{command}

\begin{command}{\pgfellipse\ooarg{drawing type}\marg{center}\marg{axis
      vector 1}\marg{axis vector 2}}
  Draws an ellipse at a given position. The drawing type can be
  |stroke|, |fill|, or |fillstroke|. 
  \example |\pgfellipse[fill]{\pgforigin}{\pgfxy(2,0)}{\pgfxy(0,1)}|
\end{command}

\begin{command}{\pgfgrid\oarg{options}\marg{lower left}\marg{upper right}}
  Draws a grid. The origin is part of the grid and the grid is clipped
  to the rectanlge specified by the \meta{lower left} and
  the \meta{upper right} corner. Allowed \meta{options} are:
  \begin{description}
  \item[\declare{|stepx=|\meta{dimension}}]
    Sets the horizontal steping to \meta{dimension}. Default is 1cm.
  \item[\declare{|stepy=|\meta{dimension}}]
    Sets the vertical steping to \meta{dimension}. Default is 1cm.
  \item[\declare{|step=|\meta{vector}}]
    Sets the horizontal stepping to the $x$-coordinate of
    \meta{vector} and the vertical steping its $y$-coordinate.
  \end{description}

  \example
\begin{verbatim}
\pgfsetlinewidth{0.8pt}
\pgfgrid[step={\pgfpoint{1cm}{1cm}}]{\pgfxy(-.3,-.3)}{\pgfxy(3.3,2.3)}{}
\pgfsetlinewidth{0.4pt}
\pgfgrid[stepx=0.1cm,stepy=0.1cm]{\pgfxy(-.15,-.15)}{\pgfxy(3.15,2.15)}
\end{verbatim}
  \begin{pgfpicture}{0cm}{0cm}{5cm}{2.5cm}
    \pgfsetlinewidth{0.8pt}
    \pgfgrid[step={\pgfpoint{1cm}{1cm}}]{\pgfxy(-.3,-.3)}{\pgfxy(3.3,2.3)}{}
    \pgfsetlinewidth{0.4pt}
    \pgfgrid[stepx=0.1cm,stepy=0.1cm]{\pgfxy(-.15,-.15)}{\pgfxy(3.15,2.15)}
  \end{pgfpicture}
\end{command}



\subsection{Image Inclusion}

The \pgf\ package offers an abstraction of the image inclusion
process, but you can still use the usual image inclusion facilities of
the |graphics| package. The main reason why you might wish to
use \pgf's image inclusion instead is that file extensions are added
automatically, depending on whether |.pdf| or |.dvi| is requested
(this is important for packages that must work with both).

The general approach to including an image is the following: First,
you use |\pgfdeclareimage| to declare the image. This must
be done prior to the first use of the image. Once you have declared an
image, you can insert it into the text using |\pgfuseimage|. The
advantage of this two-phase approach is that, at least for
\textsc{pdf}, the image data will only be included once in the
file. This can drastically reduce the file size if you use an image
repeatedly, for example in an overlay. However, there is also a
command called |\pgfimage| that declares and then immediately uses the
image.

\begin{command}{\pgfdeclareimage\oarg{options}\marg{image
      name}\marg{filename}}
  Declares an image, but does not paint anything. To draw the image,
  use |\pgfuseimage{|\meta{image name}|}|. The \meta{filename} may not
  have an extension.  For \textsc{pdf}, the extensions |.pdf|, |.jpg|,
  and |.png| will automatically tried. For PostScript, the extensions
  |.eps|, |.epsi|, and |.ps| will be tried. 

  The following options are possible:
  \begin{itemize}
  \item
    \declare{|height=|\meta{dimension}} sets the height of the
    image. If the width is not specified simultaneously, the aspect
    ratio of the image is kept.
  \item
    \declare{|width=|\meta{dimension}} sets the width of the
    image. If the height is not specified simultaneously, the aspect
    ratio of the image is kept.
  \item
    \declare{|page=|\meta{page number}} selects a given page number
    from a multipage document. Specifying this option will have the
    following effect: first, \pgf\ tries to find a file named
    \begin{quote}
      \meta{filename}|.page|\meta{page number}|.|\meta{extension}
    \end{quote}
    If such a file is found, it will be used instead of the originally
    specified filename. If not, \pgf\ inserts the image stored in
    \meta{filename}|.|\meta{extension} and if a recent version of
    |pdflatex| is used, only the selected page is inserted. For older
    versions of |pdflatex| and for |dvips| the complete document is
    inserted and a warning is printed.    
  \item
    \declare{|interpolate=|\meta{true or false}} selects whether the
    image should ``smoothed'' when zoomed. False by default.
  \item
    \declare{|mask=|\meta{mask name}} selects a transparency mask. The
    mask must previously be declared using |\pgfdeclaremask| (see
    below). This option only has an effect for |pdf|. Not all viewers
    support masking. 
  \end{itemize}
 \example
\begin{verbatim}
\pgfdeclareimage[interpolate=true,height=1cm]{image1}{pgf-tu-logo}
\pgfdeclareimage[interpolate=true,width=1cm,height=1cm]{image2}{pgf-tu-logo}
\pgfdeclareimage[interpolate=true,height=1cm]{image3}{pgf-tu-logo}
\end{verbatim}
\end{command}

\begin{command}{\pgfuseimage\marg{image name}}
  Inserts a previously declared image into the text. If you wish to
  use it in a picture environment, you should put a |\pgfbox|
  around it.

  If the macro |\pgfalternateextension| expands to some nonempty
  \meta{alternate extension}, \pgf\ will first try to use the image
  names \meta{image name}|.|\meta{alternate extension}. If this
  image is not defined, \pgf\ will next check whether \meta{alternate
    extension} contains a |!| character. If so, everythings up to this
  exclamation mark and including it is deleted from \meta{alternate
    extension} and the \pgf\ again tries to use the image \meta{image
    name}|.|\meta{alternate extension}. This is repeated until
  \meta{alternate extension} no longer contains a |!|. Then the
  original image is used.

  The |xxcolor| package sets the alternate extension to the current
  color mixin. 

  \example

\begin{verbatim}
\begin{pgfpictureboxed}{0cm}{0cm}{7cm}{2.1cm}
  \pgfputat{\pgfxy(1,1)}{\pgfbox[left,base]{\pgfuseimage{image1}}}
  \pgfputat{\pgfxy(3,1)}{\pgfbox[left,base]{\pgfuseimage{image2}}}
  \pgfputat{\pgfxy(5,1)}{\pgfbox[left,base]{\pgfuseimage{image3}}}

  \pgfrect[stroke]{\pgfxy(1,1)}{\pgfxy(1,1)}
  \pgfrect[stroke]{\pgfxy(3,1)}{\pgfxy(1,1)}
  \pgfrect[stroke]{\pgfxy(5,1)}{\pgfxy(1,1)}

  \pgfputat{\pgfxy(1,0)}{\pgfbox[left,base]{Some text.}}
\end{pgfpictureboxed}
\end{verbatim}

\pgfdeclareimage[interpolate=true,height=1cm]{image1}{pgf-tu-logo}
\pgfdeclareimage[interpolate=true,width=1cm,height=1cm]{image2}{pgf-tu-logo}
\pgfdeclareimage[interpolate=true,height=1cm]{image3}{pgf-tu-logo}

\begin{pgfpictureboxed}{0cm}{0cm}{7cm}{2.1cm}
  \pgfputat{\pgfxy(1,1)}{\pgfbox[left,base]{\pgfuseimage{image1}}}
  \pgfputat{\pgfxy(3,1)}{\pgfbox[left,base]{\pgfuseimage{image2}}}
  \pgfputat{\pgfxy(5,1)}{\pgfbox[left,base]{\pgfuseimage{image3}}}
  \pgfrect[stroke]{\pgfxy(1,1)}{\pgfxy(1,1)}
  \pgfrect[stroke]{\pgfxy(3,1)}{\pgfxy(1,1)}
  \pgfrect[stroke]{\pgfxy(5,1)}{\pgfxy(1,1)}

  \pgfputat{\pgfxy(1,0.5)}{\pgfbox[left,base]{Some text.}}
\end{pgfpictureboxed}

  The following example demonstrates the effect of using
  |\pgfuseimage| inside a color mixin environment.

\begin{verbatim}
\pgfdeclareimage[interpolate=true,height=1cm]{image1.!25!white}{pgf-tu-logo.25}
\pgfdeclareimage[interpolate=true,width=1cm,height=1cm]{image2.!25!white}{pgf-tu-logo.25}
\pgfdeclareimage[interpolate=true,height=1cm]{image3.!25!white}{pgf-tu-logo.25}
\begin{colormixin}{25!white}
\begin{pgfpictureboxed}{0cm}{0cm}{7cm}{2.1cm}
  \pgfputat{\pgfxy(1,1)}{\pgfbox[left,base]{\pgfuseimage{image1}}}
  ... % as above
\end{pgfpictureboxed}
\end{colormixin}
\end{verbatim}

\pgfdeclareimage[interpolate=true,height=1cm]{image1.!25!white}{pgf-tu-logo.25}
\pgfdeclareimage[interpolate=true,width=1cm,height=1cm]{image2.25!white}{pgf-tu-logo.25}
\pgfdeclareimage[interpolate=true,height=1cm]{image3.white}{pgf-tu-logo.25}
\begin{colormixin}{25!white}
\begin{pgfpictureboxed}{0cm}{0cm}{7cm}{2.1cm}
  \pgfputat{\pgfxy(1,1)}{\pgfbox[left,base]{\pgfuseimage{image1}}}
  \pgfputat{\pgfxy(3,1)}{\pgfbox[left,base]{\pgfuseimage{image2}}}
  \pgfputat{\pgfxy(5,1)}{\pgfbox[left,base]{\pgfuseimage{image3}}}

  \pgfrect[stroke]{\pgfxy(1,1)}{\pgfxy(1,1)}
  \pgfrect[stroke]{\pgfxy(3,1)}{\pgfxy(1,1)}
  \pgfrect[stroke]{\pgfxy(5,1)}{\pgfxy(1,1)}

  \pgfputat{\pgfxy(1,0.5)}{\pgfbox[left,base]{Some text.}}
\end{pgfpictureboxed}
\end{colormixin}
\end{command}

\begin{command}{\pgfalternateextension}
  You should redefine this command to install a different alternate
  extension.

  \example |\def\pgfalternateextension{!25!white}|
\end{command}

\begin{command}{\pgfaliasimage\marg{new image name}\marg{existing image name}}
  The \marg{existing image name} is ``cloned'' and the \marg{new image
    name} can now be used whenever original image is used. This
  command is useful for creating aliases for alternate extensions
  and for accessing the last image inserted using |\pgfimage|.
  \example |\pgfaliasimage{image.!30!white}{image.!25!white}|
\end{command}


\begin{command}{\pgfimage\oarg{options}\marg{filename}}
  Declares the image under the name |pgflastimage| and
  immediately uses it. You can ``save'' the image for later usage by
  invoking |\pgfaliasimage| on |pgflastimage|.
  \example

\begin{verbatim}
\begin{pgfpictureboxed}{0cm}{0.9cm}{7cm}{2.1cm}
  \pgfputat{\pgfxy(1,1)}{\pgfbox[left,base]
    {\pgfimage[interpolate=true,width=1cm,height=1cm]{pgf-tu-logo}}}
  \pgfputat{\pgfxy(3,1)}{\pgfbox[left,base]
    {\pgfimage[interpolate=true,width=1cm]{pgf-tu-logo}}}
  \pgfputat{\pgfxy(5,1)}{\pgfbox[left,base]
    {\pgfimage[interpolate=true,height=1cm]{pgf-tu-logo}}}

  \pgfrect[stroke]{\pgfxy(1,1)}{\pgfxy(1,1)}
  \pgfrect[stroke]{\pgfxy(3,1)}{\pgfxy(1,1)}
  \pgfrect[stroke]{\pgfxy(5,1)}{\pgfxy(1,1)}
\end{pgfpictureboxed}
\end{verbatim}

\begin{pgfpictureboxed}{0cm}{0.9cm}{7cm}{2.1cm}
  \pgfputat{\pgfxy(1,1)}{\pgfbox[left,base]{\pgfimage[interpolate=true,width=1cm,height=1cm]{pgf-tu-logo}}}
  \pgfputat{\pgfxy(3,1)}{\pgfbox[left,base]{\pgfimage[interpolate=true,width=1cm]{pgf-tu-logo}}}
  \pgfputat{\pgfxy(5,1)}{\pgfbox[left,base]{\pgfimage[interpolate=true,height=1cm]{pgf-tu-logo}}}

  \pgfrect[stroke]{\pgfxy(1,1)}{\pgfxy(1,1)}
  \pgfrect[stroke]{\pgfxy(3,1)}{\pgfxy(1,1)}
  \pgfrect[stroke]{\pgfxy(5,1)}{\pgfxy(1,1)}
\end{pgfpictureboxed}
\end{command}


\begin{command}{\pgfdeclaremask\oarg{options}\marg{mask
      name}\marg{filename}}
  Declares a transparency mask named \meta{mask name}. This mask is
  read from the file \meta{filename}. This file should contain a
  grayscale image that is as large as the actual image. A white
  pixel in the mask will correspond to ``transparent,'' a black pixel
  to ``solid,'' and grey values correspond to intermediate values. The
  image must have a single ``color channel.'' This means that the
  image must be a ``real'' grayscale image, not an \textsc{rgb}-image
  in which all \textsc{rgb}-triples happen to have the same
  components.

  The following options may be given:
  \begin{itemize}
  \item |matte=|\marg{color components} sets the so-called
    \emph{matte} of the actual image (strangely, this has to be
    specified together with the mask, not with the image itself). The
    matte is the color that has been used to preblend the image. For
    example, if the image has been preblended with a white background,
    then \meta{color components} should be set to |{1 1 1}|. White is
    the default.
  \end{itemize}
  \example
\begin{verbatim}
%% Draw a large colorful background
\pgfdeclarehorizontalshading{colorful}{5cm}{color(0cm)=(red);
color(2cm)=(green); color(4cm)=(blue); color(6cm)=(red);
color(8cm)=(green); color(10cm)=(blue); color(12cm)=(red);
color(14cm)=(green); color(16cm)=(blue)}
\hbox{\pgfuseshading{colorful}\hskip-16cm\hskip1cm
\pgfimage[height=4cm]{pgf-apple}\hskip1cm
\pgfimage[height=4cm]{pgf-apple.mask}\hskip1cm
\pgfdeclaremask{mymask}{pgf-apple.mask}
\pgfimage[mask=mymask,height=4cm,interpolate=true]{pgf-apple}}
\end{verbatim}
%% Draw a large colorful background
\pgfdeclarehorizontalshading{colorful}{5cm}{color(0cm)=(red);
color(2cm)=(green); color(4cm)=(blue); color(6cm)=(red);
color(8cm)=(green); color(10cm)=(blue); color(12cm)=(red);
color(14cm)=(green); color(16cm)=(blue)}
\hbox{\pgfuseshading{colorful}\hskip-16cm\hskip1cm
\pgfimage[height=4cm]{pgf-apple}\hskip1cm
\pgfimage[height=4cm]{pgf-apple.mask}\hskip1cm
\pgfdeclaremask{mymask}{pgf-apple.mask}
\pgfimage[mask=mymask,height=4cm,interpolate=true]{pgf-apple}}
\end{command}


To speedup the compilation, you may wish to use the following class
option:
\begin{packageoption}{{draft}}
  In draft mode boxes showing the image name replace the
  images. It is checked whether the image files exist, but they are
  not read. If either height or width is not given, 1cm is used
  instead. 
\end{packageoption}


\subsection{Text Drawing}

In order to draw text, you must use the |pgfbox| command. It
draws some text with a given alignment at the origin. Typically, you
will use a |pgfputat| to put the text at some other location
instead.

\begin{command}{\pgfbox|[|\meta{horizontal alignment}|,|\meta{vertical
      alignment}|]|\marg{\TeX\ text}}
   Draws the given text with the given alignment at the
   origin. Allowed alignments are |left|, |center|, and |right|
   horizontally; and |bottom|, |base| (the base line of the text),
  |center|, and |top| vertically.
  \example

\begin{pgfpicture}{0cm}{0cm}{13cm}{2cm}
  \pgfxyline(1,1.25)(1,0)
  \pgfputat{\pgfxy(1,1)}{\pgfbox[left,base]{left}}
  \pgfputat{\pgfxy(1,0.5)}{\pgfbox[center,base]{center}}
  \pgfputat{\pgfxy(1,0)}{\pgfbox[right,base]{right}}
  
  \pgfxyline(3,1)(12.5,1)
  \pgfputat{\pgfxy(3,1)}{\pgfbox[left,bottom]{lovely bottom}}
  \pgfputat{\pgfxy(5.5,1)}{\pgfbox[left,base]{lovely base}}
  \pgfputat{\pgfxy(8,1)}{\pgfbox[left,center]{lovely center}}
  \pgfputat{\pgfxy(10.5,1)}{\pgfbox[left,top]{lovely top}}
\end{pgfpicture}
\begin{verbatim}
  \pgfxyline(1,1.25)(1,0)
  \pgfputat{\pgfxy(1,1)}{\pgfbox[left,base]{left}}
  \pgfputat{\pgfxy(1,0.5)}{\pgfbox[center,base]{center}}
  \pgfputat{\pgfxy(1,0)}{\pgfbox[right,base]{right}}
  
  \pgfxyline(3,1)(12.5,1)
  \pgfputat{\pgfxy(3,1)}{\pgfbox[left,bottom]{lovely bottom}}
  \pgfputat{\pgfxy(5.5,1)}{\pgfbox[left,base]{lovely base}}
  \pgfputat{\pgfxy(8,1)}{\pgfbox[left,center]{lovely center}}
  \pgfputat{\pgfxy(10.5,1)}{\pgfbox[left,top]{lovely top}}
\end{verbatim}
\end{command}



\subsection{Drawing Arrows at Line Ends}

When you stroke a line or curve, \pgf\ can append arrows at the start
and at the end of the line or curve. There is a wide variety of arrows
available.

\begin{command}{\pgfsetstartarrow\marg{arrow type}}
  Henceforth, the specified \meta{arrow type} is added to all stroked 
  lines and curves. This does not apply to lines constructed using
  quick commands or lines that are stroked using |\pgfqstroke|. The
  allowed arrow types are listed below.
  \example

\begin{pgfpicture}{0cm}{0cm}{5cm}{.5cm}
  \pgfsetstartarrow{\pgfarrowto}
  \pgfsetendarrow{\pgfarrowsingle}
  \pgfxycurve(0,0.25)(0.5,0.5)(1,0)(1.5,0.25)
\end{pgfpicture}
\begin{verbatim}
  \pgfsetstartarrow{\pgfarrowto}
  \pgfsetendarrow{\pgfarrowsingle}
  \pgfxycurve(0,0.25)(0.5,0.5)(1,0)(1.5,0.25)
\end{verbatim}
\end{command}


\begin{command}{\pgfsetendarrow\marg{arrow type}}
  Like |\pgfsetstartarrow|, except that the type of
  arrow at the end is specified.
\end{command}


\begin{command}{\pgfclearstartarrow}
  Clears the setting for the start arrows.
\end{command}

\begin{command}{\pgfclearendarrow}
  Clears the setting for the end arrows.
\end{command}

The arrow types are explained below. Some arrow types take a parameter
that govern its size.

\begin{pgfpicture}{0cm}{-0.25cm}{10cm}{6cm}
  \pgfsetendarrow{\pgfarrowto}
  \pgfxyline(0,0)(1,0)
  \pgfputat{\pgfxy(1.5,0)}{\pgfbox[left,center]{\texttt{\bs pgfarrowto}}}
  
  \pgfsetendarrow{\pgfarrowsingle}
  \pgfxyline(0,0.5)(1,0.5)
  \pgfputat{\pgfxy(1.5,0.5)}{\pgfbox[left,center]{\texttt{\bs pgfarrowsingle}}}
  
  \pgfsetendarrow{\pgfarrowbar}
  \pgfxyline(0,1.0)(1,1.0)
  \pgfputat{\pgfxy(1.5,1.0)}{\pgfbox[left,center]{\texttt{\bs pgfarrowbar}}}
 
  \pgfsetendarrow{\pgfarrowsquare}
  \pgfxyline(0,1.5)(1,1.5)
  \pgfputat{\pgfxy(1.5,1.5)}{\pgfbox[left,center]{\texttt{\bs pgfarrowsquare}}}
 
  \pgfsetendarrow{\pgfarrowround}
  \pgfxyline(0,2.0)(1,2.0)
  \pgfputat{\pgfxy(1.5,2.0)}{\pgfbox[left,center]{\texttt{\bs pgfarrowround}}}
 
  \pgfsetendarrow{\pgfarrowpointed}
  \pgfxyline(0,2.5)(1,2.5)
  \pgfputat{\pgfxy(1.5,2.5)}{\pgfbox[left,center]{\texttt{\bs pgfarrowpointed}}}

  \pgfsetendarrow{\pgfarrowdot}
  \pgfxyline(0,3.0)(1,3.0)
  \pgfputat{\pgfxy(1.5,3.0)}{\pgfbox[left,center]{\texttt{\bs pgfarrowdot}}}
 
  \pgfsetendarrow{\pgfarrowdiamond}
  \pgfxyline(0,3.5)(1,3.5)
  \pgfputat{\pgfxy(1.5,3.5)}{\pgfbox[left,center]{\texttt{\bs pgfarrowdiamond}}}

  \pgfsetendarrow{\pgfarrowcircle{4pt}}
  \pgfxyline(0,4.0)(1,4.0)
  \pgfputat{\pgfxy(1.5,4.0)}{\pgfbox[left,center]{\texttt{\bs pgfarrowcirvle$\{$4pt$\}$}}}  

  \pgfsetendarrow{\pgfarrowtriangle{4pt}}
  \pgfxyline(0,4.5)(1,4.5)
  \pgfputat{\pgfxy(1.5,4.5)}{\pgfbox[left,center]{\texttt{\bs pgfarrowtriangle$\{$4pt$\}$}}}

  \pgfsetendarrow{\pgfarrowlargepointed{6pt}}
  \pgfxyline(0,5.0)(1,5.0)
  \pgfputat{\pgfxy(1.5,5.0)}{\pgfbox[left,center]{\texttt{\bs pgfarrowlargepointed$\{$6pt$\}$}}}  
\end{pgfpicture}

You can build more complicated arrow types by applying the following
modifiers.

\begin{command}{\pgfarrowswap\marg{arrow type}}
  Yields an arrow type that has a swapped direction.
  \example

\begin{pgfpicture}{0cm}{-0.25cm}{5cm}{2cm}
  \pgfsetendarrow{\pgfarrowswap{\pgfarrowto}}
  \pgfxyline(0,0)(1,0)
  \pgfputat{\pgfxy(1.5,0)}{\pgfbox[left,center]{\texttt{\bs pgfarrowswap$\{$\bs pgfarrowto$\}$}}}
  
  \pgfsetendarrow{\pgfarrowswap{\pgfarrowsingle}}
  \pgfxyline(0,0.5)(1,0.5)
  \pgfputat{\pgfxy(1.5,0.5)}{\pgfbox[left,center]{\texttt{\bs pgfarrowswap$\{$\bs pgfarrowsingle$\}$}}}
  
  \pgfsetendarrow{\pgfarrowswap{\pgfarrowbar}}
  \pgfxyline(0,1.0)(1,1.0)
  \pgfputat{\pgfxy(1.5,1.0)}{\pgfbox[left,center]{\texttt{\bs pgfarrowswap$\{$\bs pgfarrowbar$\}$}}}
 
  \pgfsetendarrow{\pgfarrowswap{\pgfarrowsquare}}
  \pgfxyline(0,1.5)(1,1.5)
  \pgfputat{\pgfxy(1.5,1.5)}{\pgfbox[left,center]{\texttt{\bs pgfarrowswap$\{$\bs pgfarrowsquare$\}$}}}
\end{pgfpicture} 
\end{command}


\begin{command}{\pgfarrowdouble\marg{arrow type}}
  Yields an arrow type that doubles the given arrow.
  \example

\begin{pgfpicture}{0cm}{-0.25cm}{5cm}{2cm}
  \pgfsetendarrow{\pgfarrowdouble{\pgfarrowto}}
  \pgfxyline(0,0)(1,0)
  \pgfputat{\pgfxy(1.5,0)}{\pgfbox[left,center]{\texttt{\bs pgfarrowdouble$\{$\bs pgfarrowto$\}$}}}
  
  \pgfsetendarrow{\pgfarrowdouble{\pgfarrowsingle}}
  \pgfxyline(0,0.5)(1,0.5)
  \pgfputat{\pgfxy(1.5,0.5)}{\pgfbox[left,center]{\texttt{\bs pgfarrowdouble$\{$\bs pgfarrowsingle$\}$}}}
  
  \pgfsetendarrow{\pgfarrowdouble{\pgfarrowbar}}
  \pgfxyline(0,1.0)(1,1.0)
  \pgfputat{\pgfxy(1.5,1.0)}{\pgfbox[left,center]{\texttt{\bs pgfarrowdouble$\{$\bs pgfarrowbar$\}$}}}
 
  \pgfsetendarrow{\pgfarrowdouble{\pgfarrowsquare}}
  \pgfxyline(0,1.5)(1,1.5)
  \pgfputat{\pgfxy(1.5,1.5)}{\pgfbox[left,center]{\texttt{\bs pgfarrowdouble$\{$\bs pgfarrowsquare$\}$}}}
\end{pgfpicture} 
\end{command}


\begin{command}{\pgfarrowtriple\marg{arrow type}}
  Yields an arrow type that triples the given arrow.
  \example

\begin{pgfpicture}{0cm}{-0.25cm}{5cm}{2cm}
  \pgfsetendarrow{\pgfarrowtriple{\pgfarrowto}}
  \pgfxyline(0,0)(1,0)
  \pgfputat{\pgfxy(1.5,0)}{\pgfbox[left,center]{\texttt{\bs pgfarrowtriple$\{$\bs pgfarrowto$\}$}}}
  
  \pgfsetendarrow{\pgfarrowtriple{\pgfarrowsingle}}
  \pgfxyline(0,0.5)(1,0.5)
  \pgfputat{\pgfxy(1.5,0.5)}{\pgfbox[left,center]{\texttt{\bs pgfarrowtriple$\{$\bs pgfarrowsingle$\}$}}}
  
  \pgfsetendarrow{\pgfarrowtriple{\pgfarrowbar}}
  \pgfxyline(0,1.0)(1,1.0)
  \pgfputat{\pgfxy(1.5,1.0)}{\pgfbox[left,center]{\texttt{\bs pgfarrowtriple$\{$\bs pgfarrowbar$\}$}}}
 
  \pgfsetendarrow{\pgfarrowtriple{\pgfarrowsquare}}
  \pgfxyline(0,1.5)(1,1.5)
  \pgfputat{\pgfxy(1.5,1.5)}{\pgfbox[left,center]{\texttt{\bs pgfarrowtriple$\{$\bs pgfarrowsquare$\}$}}}
\end{pgfpicture} 
\end{command}


\begin{command}{\pgfarrowcombine\marg{first arrow type}\marg{second
      arrow type}}
  Yields an arrow type that is made up from the two given
  arrow types, one after the other. The command
  \declare{\texttt{\string\pgfarrowcombineloose}} does the same, but gives more spacing.
  \example

\begin{pgfpicture}{0cm}{-0.25cm}{5cm}{2cm}
  \pgfsetendarrow{\pgfarrowcombine{\pgfarrowto}{\pgfarrowsingle}}
  \pgfxyline(0,0)(1,0)
  \pgfputat{\pgfxy(1.5,0)}{\pgfbox[left,center]{\texttt{\bs pgfarrowcombine$\{$\bs pgfarrowto$\}\{$\bs pgfarrowsingle$\}$}}}
  
  \pgfsetendarrow{\pgfarrowcombine{\pgfarrowsquare}{\pgfarrowround}}
  \pgfxyline(0,0.5)(1,0.5)
  \pgfputat{\pgfxy(1.5,0.5)}{\pgfbox[left,center]{\texttt{\bs pgfarrowcombine$\{$\bs pgfarrowsquare$\}\{$\bs pgfarrowround$\}$}}}
  
  \pgfsetendarrow{\pgfarrowcombine{\pgfarrowswap{\pgfarrowsingle}}{\pgfarrowsingle}}
  \pgfxyline(0,1.0)(1,1.0)
  \pgfputat{\pgfxy(1.5,1.0)}{\pgfbox[left,center]{\texttt{\bs
        pgfarrowcombine$\{$\bs pgfarrowswap$\{$\bs pgfarrowsingle$\}\}\{$\bs pgfarrowsingle$\}$}}}
 
  \pgfsetendarrow{\pgfarrowcombineloose{\pgfarrowbar}{\pgfarrowdot}}
  \pgfxyline(0,1.5)(1,1.5)
  \pgfputat{\pgfxy(1.5,1.5)}{\pgfbox[left,center]{\texttt{\bs pgfarrowcombineloose$\{$\bs pgfarrowbar$\}\{$\bs pgfarrowdot$\}$}}}
\end{pgfpicture} 
\end{command}



\subsection{Placing Labels on Lines}

Two commands can be used to place labels on lines.

\begin{command}{\pgflabel\marg{fraction}\marg{start point}\marg{end
      point}\marg{orthogonal offset}}
  This command yields a position for placing a label on a straight
  line between two points. Note that this command does not draw a
  line; it only yields a position. The \meta{offset} is orthogonal to the
  line. A \meta{fraction} of $0$ means \meta{start point}, $1$~means
  \meta{end point}, and $0.5$ means the middle.
  \example

\begin{pgfpicture}{0cm}{0cm}{5cm}{2cm}
  \pgfxyline(0,0)(5,2)
  \pgfputat{\pgflabel{.5}{\pgfxy(0,0)}{\pgfxy(5,2)}{5pt}}{\pgfcircle[stroke]{\pgforigin}{5pt}}
  \pgfputat{\pgflabel{.75}{\pgfxy(0,0)}{\pgfxy(5,2)}{5pt}}{\pgfbox[center,base]{Hi!}}
\end{pgfpicture}
\begin{verbatim}
  \pgfxyline(0,0)(5,2)
  \pgfputat
    {\pgflabel{.5}{\pgfxy(0,0)}{\pgfxy(5,2)}{1pt}}
    {\pgfcircle[stroke]{\pgforigin}{5pt}}
  \pgfputat{\pgflabel{.75}{\pgfxy(0,0)}{\pgfxy(5,2)}{5pt}}{\pgfbox[center,base]{Hi!}}
\end{verbatim}
\end{command}


\begin{command}{\pgfputlabelrotated\marg{fraction}\marg{start point}\marg{end
      point}\marg{orthogonal offset}\marg{commands}}
  This command executes the graphics commands, after having translated
  are rotated the coordinate system to the label position on a
  straight line between the two end points.
  \example

\begin{pgfpicture}{0cm}{0cm}{5cm}{2cm}
  \pgfxyline(0,0)(5,2)
  \pgfputlabelrotated{.5}{\pgfxy(0,0)}{\pgfxy(5,2)}{5pt}{\pgfcircle[stroke]{\pgforigin}{5pt}}
  \pgfputlabelrotated{.75}{\pgfxy(0,0)}{\pgfxy(5,2)}{5pt}{\pgfbox[center,base]{Hi!}}
\end{pgfpicture}
\begin{verbatim}
  \pgfxyline(0,0)(5,2)
  \pgfputlabelrotated{.5}{\pgfxy(0,0)}{\pgfxy(5,2)}{1pt}
    {\pgfcircle[stroke]{\pgforigin}{5pt}}
  \pgfputlabelrotated{.75}{\pgfxy(0,0)}{\pgfxy(5,2)}{5pt}{\pgfbox[center,base]{Hi!}}
\end{verbatim}
\end{command}



\subsection{Shadings}

The package |pgfshade| can be used to create shadings. A shading
is an area in which the color changes smoothly between different
colors. Note that you need a recent version of |pdflatex| for the
shadings to work in \textsc{pdf}. Note also that |ghostview| may do a
poor job at displaying shadings when doing anti-aliasing.

Similarly to an image, a shading must first be declared before it can
be used. Also similarly to an image, a shading is put into a
\TeX-box. Hence, in order to include a shading in a |pgfpicture|,
you have to place it in a |\pgfbox|.

There are three kinds of shadings: horizontal, vertical, and radial
shadings. However, you can rotate and clip shadings like any other
graphics object, which allows you to create more complicated
shadings. Horizontal shadings could be created by rotating a vertical
shading by 90 degrees, but explicit commands for creating both
horizontal and vertical shadings are included for convenience.

Once you have declared a shading, you can insert it into text using
the command |\pgfuseshading|.

A horizontal shading is a horizontal bar of a certain height whose
color changes smoothly. You must at least specify the colors at the
left and at the right end of the bar, but you can also add color
specifications for points in the middle. For example, suppose you
which to create a bar that is red at the left end, green in the
middle, and blue at the end. Suppose you would like the bar to be 4cm
long. This could be specified as follows:
\begin{verbatim}
rgb(0cm)=(1,0,0); rgb(2cm)=(0,1,0); rgb(4cm)=(0,0,1)
\end{verbatim}
This line means that at 0cm (the left end) of the bar, the color
should be red, which has red-green-blue (rgb) components (1,0,0). At
2cm, the bar should be green, and at 4cm it should be blue.
Instead of |rgb|, you can currently also specify |gray| as
color model, in which case only one value is needed, or |color|,
in which case you must provide the name of a color in round
brackets. In a color specification the individual specifications must
be separated using a semicolon, which may be followed by a whitespace
(like a space or a newline). Individual specifications must be given
in increasing order. 

\begin{command}{\pgfdeclarehorizontalshading\oarg{color list}\marg{shading
      name}\marg{shading height}\marg{color specification}}
  Declares a horizontal shading named \meta{shading name} of the specified
  \meta{height} with the specified colors. The length of the bar is
  automatically deduced from the maximum specification.
  \example

\pgfdeclarehorizontalshading{myshading}{1cm}{rgb(0cm)=(1,0,0); color(2cm)=(green); color(4cm)=(blue)}
\pgfuseshading{myshading}

\begin{verbatim}
\pgfdeclarehorizontalshading{myshading}{1cm}%
  {rgb(0cm)=(1,0,0); color(2cm)=(green); color(4cm)=(blue)}
\pgfuseshading{myshading}
\end{verbatim}

  The effect of the \meta{color list}, which is a
  comma-separated list of colors, is the following: Normally, when
  this list is empty, once a shading is declared it becomes
  ``frozen.'' This means that even if you change a color that was used
  in the declaration of the shading later on, the shading will not
  change. By specifying a \meta{color list} you can specify
  that the shading should be recalculated whenever one of the colors
  listed in the list changes (this includes effects like color
  mixins). Thus, when you specify a \meta{color list},
  whenever the shading is used, \pgf\ first converts the colors in the
  list to \textsc{rgb} triples using the current values of the
  colors and taking any mixins and blendings into account. If the
  resulting \textsc{rgb} triples have not yet been   used, a new
  shading is internally created and used. Note that if the 
  option \meta{color list} is used, then no shading is created until
  the first use of |\pgfuseshading|. In particular, the colors
  mentioned in the shading need not be defined when the declaration is
  given.

  When a shading is recalculated because of a change in the
  colors mentioned in \meta{color list}, the complete shading
  is recalculated. Thus even colors not mentioned in the list will be
  used with their current values, not with the values they had upon
  declaration. 
  \example
\pgfdeclarehorizontalshading[mycolor]{myshading}{1cm}{rgb(0cm)=(1,0,0); color(2cm)=(mycolor)}
\colorlet{mycolor}{green}
\pgfuseshading{myshading}
\colorlet{mycolor}{blue}
\pgfuseshading{myshading}

\begin{verbatim}
\pgfdeclarehorizontalshading[mycolor]{myshading}{1cm}{rgb(0cm)=(1,0,0); color(2cm)=(mycolor)}
\colorlet{mycolor}{green}
\pgfuseshading{myshading}
\colorlet{mycolor}{blue}
\pgfuseshading{myshading}
\end{verbatim}
\end{command}


\begin{command}{\pgfdeclareverticalshading\marg{shading
      name}\marg{shading width}\marg{color specification}}
   Declares a vertical shading named \meta{shading name} of the
   specified \meta{width}. The height of the bar is automatically
   deduced from the maximum specification.
  \example

\pgfdeclareverticalshading{myshading2}{5cm}{rgb(0cm)=(1,0,0); rgb(1.5cm)=(0,1,0); rgb(2cm)=(0,0,1)}
\pgfuseshading{myshading2}

\begin{verbatim}
\pgfdeclareverticalshading{myshading}{5cm}%
  {rgb(0cm)=(1,0,0); rgb(1.5cm)=(0,1,0); rgb(2cm)=(0,0,1)}
\pgfuseshading{myshading}
\end{verbatim}
\end{command}


\begin{command}{\pgfdeclareradialshading\marg{shading
      name}\marg{center point}\marg{color specification}}
  Declares an radial shading. A radial shading is a circle whose inner
  color changes as specified by the color specification. Assuming that
  the center of the shading is at the origin, the color of the center
  will be the color specified for 0cm and the color of the border of
  the circle will be the color for the maximum specification. The
  radius of the circle will be the maximum specification. If the
  center coordinate is not at the origin, the whole shading inside the
  circle (whose size remains exactly the same) will be distorted such
  that the given center now has the color specified for 0cm.
  \example

\pgfdeclareradialshading{sphere}{\pgfpoint{0.5cm}{0.5cm}}%
 {rgb(0.3cm)=(0.9,0,0); rgb(0.7cm)=(0.7,0,0); rgb(1cm)=(0.5,0,0); rgb(1.05cm)=(1,1,1)}
\pgfuseshading{sphere}

\begin{verbatim}
\pgfdeclareradialshading{sphere}{\pgfpoint{0.5cm}{0.5cm}}%
 {rgb(0.3cm)=(0.9,0,0);
  rgb(0.7cm)=(0.7,0,0);
  rgb(1cm)=(0.5,0,0);
  rgb(1.05cm)=(1,1,1)}
\pgfuseshading{sphere}
\end{verbatim}
\end{command}

\begin{command}{\pgfuseshading\marg{shading name}}
  Inserts a previously declared shading into the text. If you wish to
  use it in a |pgfpicture| environment, you should put a |\pgfbox|
  around it. Like |\pgfuseimage|, alternate extensions are tried
  before the actual shading is used.
  \example

\begin{verbatim}
  \pgfputat{\pgfxy(1,1)}{\pgfbox[center,center]{\pgfuseshading{myshading}}}
\end{verbatim}
\end{command}


\begin{command}{\pgfaliasshading\marg{new shading name}\marg{existing shading name}}
  The \meta{existing shading name} is ``cloned'' and the shading
  \meta{new shading name} can now be used whenever original shading
  is used. This command is mainly useful for creating aliases for
  environments that use alternate extensions.
  \example \verb/\pgfaliasshading{shading!30}{shading!25}/
\end{command}



\section{Using Nodes}

The package |pgfnodes| allows you to draw all sorts of graphs
in a convenient way. You draw them by first defining
\emph{nodes}. Once you have defined a node, you can connect nodes
using lines or curves. The advantage of using nodes is that if, later
on, you decide to move a node slightly, all connecting lines `follow'
automatically.



\subsection{Node Creation}

In all of the following command, the possible drawing types are
|stroke|, |fill|, |fillstroke|, and |virtual| (draws nothing).

\begin{command}{\pgfnodecircle\marg{node name}\ooarg{drawing type}\marg{center}\marg{radius}}
  Creates a circular node with the given radius at the
  given position.
  \example

\begin{pgfpicture}{0cm}{0cm}{5cm}{2cm}
  \pgfnodecircle{Node1}[stroke]{\pgfxy(1,1)}{0.5cm}
  \pgfnodecircle{Node2}[virtual]{\pgfxy(3,0.5)}{0.25cm}
  \pgfnodecircle{Node3}[fill]{\pgfxy(5,1)}{0.25cm}

  \pgfnodeconnline{Node1}{Node2}
  \pgfnodeconnline{Node2}{Node3}
\end{pgfpicture}
\begin{verbatim}
  \pgfnodecircle{Node1}[stroke]{\pgfxy(1,1)}{0.5cm}
  \pgfnodecircle{Node2}[virtual]{\pgfxy(3,0.5)}{0.25cm}
  \pgfnodecircle{Node3}[fill]{\pgfxy(5,1)}{0.25cm}

  \pgfnodeconnline{Node1}{Node2}
  \pgfnodeconnline{Node2}{Node3}
\end{verbatim}
\end{command}



\begin{command}{\pgfnoderect\marg{node name}\ooarg{drawing
      type}\marg{center}\marg{width/height vector}}
  Creates a rectangular node with the width and height that
  is centered at the given position.
  \example

\begin{pgfpicture}{0cm}{0cm}{5cm}{2cm}
  \pgfnoderect{Node1}[fill]{\pgfxy(1,1)}{\pgfxy(1,0.5)}
  \pgfnodecircle{Node2}[virtual]{\pgfxy(3,0.5)}{0.25cm}
  \pgfnoderect{Node3}[stroke]{\pgfxy(5,1)}{\pgfxy(2,1)}

  \pgfnodeconnline{Node1}{Node2}
  \pgfnodeconnline{Node2}{Node3}
\end{pgfpicture}
\begin{verbatim}
  \pgfnoderect{Node1}[fill]{\pgfxy(1,1)}{\pgfxy(1,0.5)}
  \pgfnodecircle{Node2}[virtual]{\pgfxy(3,0.5)}{0.25cm}
  \pgfnoderect{Node3}[stroke]{\pgfxy(5,1)}{\pgfxy(2,1)}

  \pgfnodeconnline{Node1}{Node2}
  \pgfnodeconnline{Node2}{Node3}
\end{verbatim}
\end{command}


\begin{command}{\pgfnodebox\marg{node name}\ooarg{drawing
      type}\marg{center}\marg{\TeX\ text}\marg{horiz.\
      margin}\marg{vert.\ margin}}
  Creates a rectangular node that is centered at \meta{center}. The
  size of the node is calculated from the size of the box that is
  placed inside. The margins can be used to leave a little space
  around the text. 
  \example

\begin{pgfpicture}{0cm}{0cm}{5cm}{2cm}
  \pgfnodebox{Node1}[stroke]{\pgfxy(1,1)}{Hi!}{2pt}{2pt}
  \pgfnodebox{Node2}[virtual]{\pgfxy(3,0.5)}{There}{2pt}{2pt}
  \pgfnodebox{Node3}[stroke]{\pgfxy(5,1)}{You}{10pt}{0pt}

  \pgfnodeconnline{Node1}{Node2}
  \pgfnodeconnline{Node2}{Node3}
\end{pgfpicture}
\begin{verbatim}
  \pgfnodebox{Node1}[stroke]{\pgfxy(1,1)}{Hi!}{2pt}{2pt}
  \pgfnodebox{Node2}[virtual]{\pgfxy(3,0.5)}{There}{2pt}{2pt}
  \pgfnodebox{Node3}[stroke]{\pgfxy(5,1)}{You}{10pt}{0pt}

  \pgfnodeconnline{Node1}{Node2}
  \pgfnodeconnline{Node2}{Node3}
\end{verbatim}
\end{command}



\subsection{Coordinates Relative to Nodes}

\begin{command}{\pgfnodecenter\marg{node name}}
  Yields the center of a node. This command is especially
  useful for placing nodes relative to other nodes.
  \example

\begin{pgfpicture}{0cm}{.25cm}{5cm}{1cm}
  \pgfnodecircle{Node1}[stroke]{\pgfxy(1,0.5)}{0.25cm}
  \pgfnodecircle{Node2}[stroke]
    {\pgfrelative{\pgfxy(1,0)}{\pgfnodecenter{Node1}}}{0.25cm}
  \pgfnodecircle{Node3}[stroke]
    {\pgfrelative{\pgfxy(1,0)}{\pgfnodecenter{Node2}}}{0.25cm}
  \pgfnodecircle{Node4}[stroke]
    {\pgfrelative{\pgfxy(1,0)}{\pgfnodecenter{Node3}}}{0.25cm}
\end{pgfpicture}
\begin{verbatim}
  \pgfnodecircle{Node1}[stroke]{\pgfxy(1,0.5)}{0.25cm}
  \pgfnodecircle{Node2}[stroke]
    {\pgfrelative{\pgfxy(1,0)}{\pgfnodecenter{Node1}}}{0.25cm}
  \pgfnodecircle{Node3}[stroke]
    {\pgfrelative{\pgfxy(1,0)}{\pgfnodecenter{Node2}}}{0.25cm}
  \pgfnodecircle{Node4}[stroke]
    {\pgfrelative{\pgfxy(1,0)}{\pgfnodecenter{Node3}}}{0.25cm}
\end{verbatim}
\end{command}


\begin{command}{\pgfnodeborder\marg{node name}\marg{angle}\marg{border
      offset}}
  Returns a position on the border of the node named \meta{node
    name} at an angle of \meta{angle} (in degrees). For a positive
  offset, the position is removed from the border by the amount of the
  offset.
  \example

\begin{pgfpicture}{0cm}{.25cm}{5cm}{1cm}
  \pgfnodebox{Node1}[stroke]{\pgfxy(1,0.5)}{hello world}{2pt}{2pt}

  \pgfcircle[fill]{\pgfnodeborder{Node1}{0}{5pt}}{2pt}
  \pgfcircle[fill]{\pgfnodeborder{Node1}{10}{5pt}}{2pt}
  \pgfcircle[fill]{\pgfnodeborder{Node1}{20}{5pt}}{2pt}
  \pgfcircle[fill]{\pgfnodeborder{Node1}{30}{5pt}}{2pt}
  \pgfcircle[fill]{\pgfnodeborder{Node1}{40}{5pt}}{2pt}
  \pgfcircle[fill]{\pgfnodeborder{Node1}{50}{5pt}}{2pt}
  \pgfcircle[fill]{\pgfnodeborder{Node1}{60}{5pt}}{2pt}
\end{pgfpicture}
\begin{verbatim}
  \pgfnodebox{Node1}[stroke]{\pgfxy(1,0.5)}{hello world}{2pt}{2pt}

  \pgfcircle[fill]{\pgfnodeborder{Node1}{0}{5pt}}{2pt}
  \pgfcircle[fill]{\pgfnodeborder{Node1}{10}{5pt}}{2pt}
  \pgfcircle[fill]{\pgfnodeborder{Node1}{20}{5pt}}{2pt}
  \pgfcircle[fill]{\pgfnodeborder{Node1}{30}{5pt}}{2pt}
  \pgfcircle[fill]{\pgfnodeborder{Node1}{40}{5pt}}{2pt}
  \pgfcircle[fill]{\pgfnodeborder{Node1}{50}{5pt}}{2pt}
  \pgfcircle[fill]{\pgfnodeborder{Node1}{60}{5pt}}{2pt}
\end{verbatim}
\end{command}


\begin{command}{\pgfconnstart\ooarg{border offset}\marg{start
      node}\marg{end node}}
  Returns a position on the border of the first node for a
  line in the direction of the second node.
  \example

\begin{pgfpicture}{0cm}{0cm}{5cm}{1.25cm}
  \pgfnodebox{Node1}[stroke]{\pgfxy(1,0.5)}{hello world}{2pt}{2pt}
  \pgfnodebox{Node2}[stroke]{\pgfxy(3,0)}{2}{2pt}{2pt}
  \pgfnodebox{Node3}[stroke]{\pgfxy(3,0.5)}{3}{2pt}{2pt}
  \pgfnodebox{Node4}[stroke]{\pgfxy(3,1)}{4}{2pt}{2pt}
  
  \pgfcircle[fill]{\pgfnodeconnstart[5pt]{Node1}{Node2}}{2pt}
  \pgfcircle[fill]{\pgfnodeconnstart[10pt]{Node1}{Node3}}{2pt}
  \pgfcircle[fill]{\pgfnodeconnstart[15pt]{Node1}{Node4}}{2pt}
\end{pgfpicture}
\begin{verbatim}
  \pgfnodebox{Node1}[stroke]{\pgfxy(1,0.5)}{hello world}{2pt}{2pt}
  \pgfnodebox{Node2}[stroke]{\pgfxy(3,0)}{2}{2pt}{2pt}
  \pgfnodebox{Node3}[stroke]{\pgfxy(3,0.5)}{3}{2pt}{2pt}
  \pgfnodebox{Node4}[stroke]{\pgfxy(3,1)}{4}{2pt}{2pt}
  
  \pgfcircle[fill]{\pgfnodeconnstart[5pt]{Node1}{Node2}}{2pt}
  \pgfcircle[fill]{\pgfnodeconnstart[10pt]{Node1}{Node3}}{2pt}
  \pgfcircle[fill]{\pgfnodeconnstart[15pt]{Node1}{Node4}}{2pt}
\end{verbatim}
\end{command}




\subsection{Connecting Nodes}

\begin{command}{\pgfnodesetsepstart\marg{offset}}
  Sets the offset for the start of lines that are drawn
  using the below node connection commands. Use
  \declare{\texttt{\string\pgfnodesetsepend}} for setting the end
  offset.
  \example
  
\begin{pgfpicture}{0cm}{0.25cm}{5cm}{1.25cm}
  \pgfnodebox{Node1}[stroke]{\pgfxy(1,0.5)}{hello world}{2pt}{2pt}
  \pgfnodebox{Node2}[stroke]{\pgfxy(4,.5)}{2}{2pt}{2pt}

  \pgfnodesetsepstart{0pt}
  \pgfnodesetsepend{5pt}
  \pgfsetendarrow{\pgfarrowto}

  \pgfnodeconnline{Node1}{Node2}
\end{pgfpicture}
\begin{verbatim}
  \pgfnodebox{Node1}[stroke]{\pgfxy(1,0.5)}{hello world}{2pt}{2pt}
  \pgfnodebox{Node2}[stroke]{\pgfxy(4,.5)}{2}{2pt}{2pt}

  \pgfnodesetsepstart{0pt}
  \pgfnodesetsepend{5pt}
  \pgfsetendarrow{\pgfarrowto}

  \pgfnodeconnline{Node1}{Node2}
\end{verbatim}
\end{command}


\begin{command}{\pgfnodeconnline\marg{start node}\marg{end node}}
  Draws a straight line from the border of the first node
  to the border of the second node.
  \example |\pgfnodeconnline{A}{B}|
\end{command}


\begin{command}{\pgfnodeconncurve\marg{start node}\marg{end node}%
    \marg{start angle}\marg{end angle}\marg{$d_1$}\marg{$d_2$}}
  Draws a curve from the \meta{start node} to the \meta{end node}. The
  curve will start at the \meta{start angle} on the border of the
  \meta{start node}. It ends at angle \meta{end angle}  on the border
  of the \marg{end node}. The parameters $d_1$ and $d_2$ are the
  distances of the first, respectively second, support point from the
  border of the first, respectively second, node.
  \example

\begin{pgfpicture}{0cm}{0cm}{9cm}{2cm}
  \pgfnodebox{Node1}[stroke]{\pgfxy(1,0.5)}{hello}{2pt}{2pt}
  \pgfnodebox{Node2}[stroke]{\pgfxy(4,.5)}{world}{2pt}{2pt}
  \pgfnodebox{Node3}[stroke]{\pgfxy(7,.5)}{lovely}{2pt}{2pt}

  \pgfnodeconncurve{Node1}{Node2}{0}{90}{1cm}{1cm}
  \pgfnodeconncurve{Node1}{Node2}{0}{90}{1cm}{1.5cm}
  \pgfnodeconncurve{Node1}{Node2}{0}{90}{1cm}{2cm}
  \pgfnodeconncurve{Node1}{Node2}{0}{90}{1cm}{2.5cm}
  
  \pgfnodeconncurve{Node2}{Node3}{-10}{80}{1cm}{1cm}
  \pgfnodeconncurve{Node2}{Node3}{-20}{70}{1cm}{1cm}
  \pgfnodeconncurve{Node2}{Node3}{-30}{60}{1cm}{1cm}
  \pgfnodeconncurve{Node2}{Node3}{-40}{50}{1cm}{1cm}
\end{pgfpicture}
\begin{verbatim}
  \pgfnodebox{Node1}[stroke]{\pgfxy(1,0.5)}{hello}{2pt}{2pt}
  \pgfnodebox{Node2}[stroke]{\pgfxy(4,.5)}{world}{2pt}{2pt}
  \pgfnodebox{Node3}[stroke]{\pgfxy(7,.5)}{lovely}{2pt}{2pt}

  \pgfnodeconncurve{Node1}{Node2}{0}{90}{1cm}{1cm}
  \pgfnodeconncurve{Node1}{Node2}{0}{90}{1cm}{1.5cm}
  \pgfnodeconncurve{Node1}{Node2}{0}{90}{1cm}{2cm}
  \pgfnodeconncurve{Node1}{Node2}{0}{90}{1cm}{2.5cm}
  
  \pgfnodeconncurve{Node2}{Node3}{-10}{80}{1cm}{1cm}
  \pgfnodeconncurve{Node2}{Node3}{-20}{70}{1cm}{1cm}
  \pgfnodeconncurve{Node2}{Node3}{-30}{60}{1cm}{1cm}
  \pgfnodeconncurve{Node2}{Node3}{-40}{50}{1cm}{1cm}
\end{verbatim}
\end{command}



\subsection{Placing Labels on Node Connections}

\begin{command}{\pgfnodelabel\marg{start node}\marg{end node}%
    \marg{fraction}\ooarg{vertical offset}\marg{command}}
  This command places a label at the given \meta{fraction} of a
  straight line between two nodes.
  \example

\begin{pgfpicture}{0cm}{0cm}{5cm}{2cm}
  \pgfnodebox{Node1}[stroke]{\pgfxy(1,0.5)}{hello}{2pt}{2pt}
  \pgfnodebox{Node2}[stroke]{\pgfxy(5,1.5)}{world}{2pt}{2pt}

  \pgfnodeconnline{Node1}{Node2}
  \pgfnodelabel{Node1}{Node2}[0.5][5pt]{\pgfcircle[stroke]{\pgforigin}{5pt}}
  \pgfnodelabel{Node1}{Node2}[0.75][5pt]{\pgfbox[center,base]{Hi!}}
\end{pgfpicture}
\begin{verbatim}
  \pgfnodebox{Node1}[stroke]{\pgfxy(1,0.5)}{hello}{2pt}{2pt}
  \pgfnodebox{Node2}[stroke]{\pgfxy(5,1.5)}{world}{2pt}{2pt}

  \pgfnodeconnline{Node1}{Node2}
  \pgfnodelabel{Node1}{Node2}[0.5][5pt]{\pgfcircle[stroke]{\pgforigin}{5pt}}
  \pgfnodelabel{Node1}{Node2}[0.75][5pt]{\pgfbox[center,base]{Hi!}}
\end{verbatim}
\end{command}

\begin{command}{\pgfnodelabelrotated\marg{start node}\marg{end node}%
    \marg{fraction}\ooarg{vertical offset}\marg{command}}
  This command places a rotated label at the given \meta{fraction} of
  a straight line between two nodes. The label is rotated according to the slope
  of the line.
  \example

\begin{pgfpicture}{0cm}{0cm}{5cm}{2cm}
  \pgfnodebox{Node1}[stroke]{\pgfxy(1,0.5)}{hello}{2pt}{2pt}
  \pgfnodebox{Node2}[stroke]{\pgfxy(5,1.5)}{world}{2pt}{2pt}

  \pgfnodeconnline{Node1}{Node2}
  \pgfnodelabelrotated{Node1}{Node2}[0.5][5pt]{\pgfcircle[stroke]{\pgforigin}{5pt}}
  \pgfnodelabelrotated{Node1}{Node2}[0.75][5pt]{\pgfbox[center,base]{Hi!}}
\end{pgfpicture}
\begin{verbatim}
  \pgfnodebox{Node1}[stroke]{\pgfxy(1,0.5)}{hello}{2pt}{2pt}
  \pgfnodebox{Node2}[stroke]{\pgfxy(5,1.5)}{world}{2pt}{2pt}

  \pgfnodeconnline{Node1}{Node2}
  \pgfnodelabelrotated{Node1}{Node2}[0.5][5pt]{\pgfcircle[stroke]{\pgforigin}{5pt}}
  \pgfnodelabelrotated{Node1}{Node2}[0.75][5pt]{\pgfbox[center,base]{Hi!}}
\end{verbatim}
\end{command}



\section{Extended Color Support}

This section documents the package \texttt{xxcolor}, which is
currently distributed as part of \pgf. This package extends the
\texttt{xcolor} package, written by Uwe Kern, which in turn extends
the \texttt{color} package. I hope that the commands in
\texttt{xxcolor} will some day migrate to \texttt{xcolor}, such that
this package becomes superfluous.

The main aim of the \texttt{xxcolor} package is to provide an
environment inside which all colors are ``washed out'' or ``dimmed.''
This is useful in numerous situations and must typically be achieved
in a roundabout manner if such an environment is not available.

\begin{environment}{{colormixin}\marg{mix-in specification}}
  The mix-in specification is applied to all colors inside
  the environment. At the beginning of the environment, the mix-in is
  applied to the current color, i.\,e., the color that was in effect
  before the environment started. A mix-in specification is a number
  between 0 and 100 followed by an exclamation mark and a color
  name. When a |\color| command is 
  encountered inside a mix-in environment, the number states what
  percentage of the desired color should be used. The rest is
  ``filled up'' with the color given in the mix-in
  specification. Thus, a mix-in specification like |90!blue|
  will mix in 10\% of blue into everything, whereas |25!white| will
  make everything nearly white.
  \example
\begin{verbatim}
\color{red}Red text,%
\begin{colormixin}{25!white}
  washed-out red text,
  \color{blue} washed-out blue text,
  \begin{colormixin}{25!black}
    dark washed-out blue text,
    \color{green} dark washed-out green text,%
  \end{colormixin}
  back to washed-out blue text,%
\end{colormixin}
and back to red.
\end{verbatim}

{
\noindent\color{red}Red text,
\begin{colormixin}{50!white}
  washed-out red text,
  \color{blue} washed-out blue text,
  \begin{colormixin}{25!black}
    dark washed-out blue text,
    \color{green} dark washed-out green text,%
  \end{colormixin}
  back to washed-out blue text,%
\end{colormixin}
and back to red.}
\end{environment}

Note that the environment only changes colors that have been installed
using the standard \LaTeX\ |\color| command. In particular,
the colors in images are not changed. There is, however, some support
offered by the commands |\pgfuseimage| and
|\pgfuseshading|. If the first command is invoked 
inside a |colormixin| environment with the parameter, say,
|50!black| on an image with the name |foo|, the command
will first check whether there is also a defined image with the name
|foo.!50!black|. If so, this image is used instead. This allows
you to provide a different image for this case. If you nest
|colormixin| environments, the different mix-ins are appended as
a comma-separated list. For example, inside the inner environment of
the above example, |\pgfuseimage{foo}| would first check whether
there exists an image named |foo.!50!white!25!black|.

\begin{command}{\colorcurrentmixin}
  Expands to the current accumulated mix-in. Each nesting of a
  |colormixin| adds a mix-in to this list.
  \example 
\begin{verbatim}
\begin{colormixin}{25!white}
  \colorcurrentmixin is now ``25!white''
  \begin{colormixin}{75!black}
    \colorcurrentmixin is now ``75!black!25!white''
    \begin{colormixin}{50!white}
      \colorcurrentmixin is now ``50!white!75!black!25!white''
    \end{colormixin}
  \end{colormixin}
\end{colormixin}
\end{verbatim}
\end{command}




\end{document}


